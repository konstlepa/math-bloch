\usepackage{stix}
\usepackage[utf8]{inputenc}
\usepackage[english]{babel}
\usepackage{xcolor}
\usepackage{enumitem}
\usepackage{mdframed}
\usepackage[hidelinks]{hyperref}
\usepackage{thmtools}
\usepackage{thm-restate}

\mdfsetup{
	innerleftmargin=6pt,
	innerrightmargin=6pt,
	splittopskip=\topskip,
	skipabove=1.5em
}

\makeatletter
\newcommand\xlabel[2][]{\phantomsection\def\@currentlabelname{#1}\label{#2}}
\makeatother

\usepackage[active,tightpage]{preview}
\usepackage{preview}

\renewcommand{\PreviewBorder}{0.5em}

%% Restore par-indent for preview.
\usepackage{etoolbox}
\edef\keptparindent{\the\parindent}
\patchcmd{\preview}
{\ignorespaces} %%% \preview ends with \ignorespaces
{\parindent\keptparindent\ignorespaces}
{}{}

\newcommand{\Newpage}{\end{preview}\begin{preview}}

\makeatletter
\def\p@section{\thechapter.}
\makeatother

\setlist[1]{leftmargin=*,labelindent=\parindent}

\newlist{bycases}{enumerate}{10}
\setlist[bycases]{leftmargin=*}
\setlist[bycases,1]{label=Case \arabic*., ref=\arabic*}
\setlist[bycases,2]{label=Case \theenumi\,\alph*., ref=\theenumi.\alph*}

\newlist{lenumerate}{enumerate}{10}
\setlist[lenumerate]{label=(\alph*), leftmargin=*}
\setlist[lenumerate,1]{ref=\alph*}
\setlist[lenumerate,2]{ref=\theenumi \, \alph*}

\setlist[enumerate]{leftmargin=*}
\setlist[enumerate,1]{label=(\arabic*), ref=\arabic*}
\setlist[enumerate,2]{label=(\alph*), ref=\theenumi \, \alph*}

\renewcommand{\datename}{}

\declaretheoremstyle[
	headfont=\normalfont\bfseries,
	notefont=\normalfont\bfseries,
	notebraces={(}{)},
	bodyfont=\normalfont,
	postheadspace=0.5em,
	preheadhook=\begin{mdframed}, postfoothook=\end{mdframed}
]{exercisestyle}

\declaretheoremstyle[
	headfont=\normalfont\bfseries,
	notefont=\normalfont\bfseries,
	notebraces={(}{)},
	bodyfont=\normalfont\itshape,
	postheadspace=0.5em,
	spaceabove=1.5em
]{theoremstyle}

\declaretheorem[style=exercisestyle, numberwithin=section]{exercise}
\declaretheorem[style=theoremstyle, numberwithin=section]{theorem}
\declaretheorem[style=theoremstyle, sibling=theorem]{lemma}
\declaretheorem[style=theoremstyle, sibling=theorem]{definition}
\declaretheorem[style=theoremstyle, sibling=theorem]{axiom}
\declaretheorem[style=theoremstyle, sibling=theorem]{corollary}

\renewcommand{\theexercise}{\thechapter.\thesection.\arabic{exercise}}
\renewcommand{\thetheorem}{\thechapter.\thesection.\arabic{theorem}}
\renewcommand{\thelemma}{\thechapter.\thesection.\arabic{lemma}}
\renewcommand{\thedefinition}{\thechapter.\thesection.\arabic{definition}}
\renewcommand{\thecorollary}{\thechapter.\thesection.\arabic{corollary}}
\renewcommand{\theaxiom}{\thechapter.\thesection.\arabic{axiom}}

\let\proof\relax
\declaretheoremstyle[
	spaceabove=6pt,
	spacebelow=6pt,
	headfont=\normalfont\itshape\bfseries,
	notefont=\normalfont\itshape\bfseries,
	notebraces={of }{},
	bodyfont=\normalfont,
	postheadspace=1em,
	qed=$\square$
]{proofstyle}
\declaretheorem[style=proofstyle, unnumbered]{proof}

\newcommand{\bibbloch}[1]{
	\bibitem{bloch#1}
	Ethan~D. Bloch,
	\emph{The Real Numbers and Real Analysis},
	1st ed.,
	Springer,
	2011.
}

\newcommand{\TBD}{{\color{red} TBD}}
\newcommand{\TODO}{{\color{red} TODO}}

\newenvironment{notmine}[1][]{\color{gray}}{}

\let\emph\relax
\DeclareTextFontCommand{\emph}{\bfseries}