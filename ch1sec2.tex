%=============================================================================================================
%= SECTION 1.2
%=============================================================================================================
\section{Entry 1: Axioms for the Natural Numbers}
\label{nat}

\begin{axiom}[Peano Postulates] %% 1.2.1
	\label{nat:a:peano}
	There exists a set $\mathbb{N}$ with an element $1 \in \mathbb{N}$ and a function $s : \mathbb{N} \to \mathbb{N}$ that satisfy the following three properties.
	\begin{lenumerate}
		\item \label{nat:a:peano:1}
		      There is no $n \in \mathbb{N}$ such that $s(n) = 1$.
		\item \label{nat:a:peano:injective}
		      The function $s$ is injective.
		\item \label{nat:a:peano:induction}
		      Let $G \subseteq \mathbb{N}$ be a set. Suppose that $1 \in G$, and that if $g \in G$ then $s(g) \in G$. Then
		      $G = \mathbb{N}$.
	\end{lenumerate}
\end{axiom}

\begin{definition} %% 1.2.2
	\label{nat:d:natural}
	The set of \emph{natural numbers}, denoted $\mathbb{N}$, is the set the existence of which is given in the \nameref{nat:a:peano}.
\end{definition}

\begin{lemma} %% 1.2.3
	\label{nat:l:exists_prev}
	Let $a \in \mathbb{N}$. Suppose that $a \not= 1$. Then there is a unique $b \in \mathbb{N}$ such that $a = s(b)$.
\end{lemma}

\begin{theorem}[Definition by Recursion] %% 1.2.4
	\label{nat:t:recusion}
	Let $H$ be a set, let $e \in H$ and let $k : H \to H$ be a function. Then there is a unique function $f : \mathbb{N} \to H$ such that $f(1) = e$, and that $f \circ s = k \circ f$.
\end{theorem}

\begin{theorem} %% 1.2.5
	\label{nat:t:add}
	There is a unique binary operation $+ : \mathbb{N} \times \mathbb{N} \to \mathbb{N}$ that satisfies the following two properties for all $n, m \in \mathbb{N}$.
	\begin{lenumerate}
		\item \label{nat:t:add:1}
		      $n + 1 = s(n)$.
		\item \label{nat:t:add:n}
		      $n + s(m) = s(n + m)$.
	\end{lenumerate}
\end{theorem}

\begin{theorem} %% 1.2.6
	\label{nat:t:mult}
	There is a unique binary operation $\cdot : \mathbb{N} \times \mathbb{N} \to \mathbb{N}$ that satisfies the
	following two properties for all $n, m \in \mathbb{N}$.
	\begin{lenumerate}
		\item \label{nat:t:mult:1}
		      $n \cdot 1 = n$.
		\item \label{nat:t:mult:n}
		      $n \cdot s(m) = (n \cdot m) + n$.
	\end{lenumerate}
\end{theorem}

\begin{theorem} %% 1.2.7
	\label{nat:t:props}
	Let $a,b,c \in \mathbb{N}$.
	\begin{enumerate}
		\item \xlabel[Cancellation Law for Addition]{nat:t:props:cancellation_add}
		      If $a + c = b + c$, then $a = b$ \quad (\nameref*{nat:t:props:cancellation_add}).
		\item \xlabel[Associative Law for Addition]{nat:t:props:associative_add}
		      $(a + b) + c = a + (b + c)$ \quad (\nameref*{nat:t:props:associative_add}).
		\item \label{nat:t:props:3}
		      $1 + a = s(a) = a + 1$.
		\item \xlabel[Commutative Law for Addition]{nat:t:props:commutative_add}
		      $a + b = b + a$ \quad (\nameref*{nat:t:props:commutative_add}).
		\item \label{nat:t:props:5}
		      $a + b \not= 1$.
		\item \label{nat:t:props:6}
		      $a + b \not= a$.
		\item \xlabel[Identity Law for Multiplication]{nat:t:props:identity_mult}
		      $a \cdot 1 = a = 1 \cdot a$ \quad (\nameref*{nat:t:props:identity_mult}).
		\item \xlabel[Distributive Law]{nat:t:props:distributive_right}
		      $(a + b)c = a c + b c$ \quad (\nameref*{nat:t:props:distributive_right}).
		\item \xlabel[Commutative Law for Multiplication]{nat:t:props:commutative_mult}
		      $a b = b a$ \quad (\nameref*{nat:t:props:commutative_mult}).
		\item \xlabel[Distributive Law]{nat:t:props:distributive_left}
		      $c(a + b) = c a + c b$ \quad (\nameref*{nat:t:props:distributive_left}).
		\item \xlabel[Associative Law for Multiplication]{nat:t:props:associative_mult}
		      $(a b)c = a(b c)$ \quad (\nameref*{nat:t:props:associative_mult}).
		\item \xlabel[Cancellation Law for Multiplication]{nat:t:props:cancellation_mult}
		      If $a c = b c$ then $a = b$ \quad (\nameref*{nat:t:props:cancellation_mult}).
		\item \label{nat:t:props:13}
		      $a b = 1$ if and only if $a = 1 = b$.
	\end{enumerate}
\end{theorem}

\begin{definition} %% 1.2.8
	\label{nat:d:relation}
	The relation $<$ on $\mathbb{N}$ is defined by $a < b$ if and only if there is some $p \in \mathbb{N}$ such that $a + p = b$, for all $a, b \in \mathbb{N}$. The relation $\leq$ on $\mathbb{N}$ is defined by $a \leq b$ if and only if $a < b$ or $a = b$, for all $a,b \in \mathbb{N}$.
\end{definition}

\begin{theorem} %% 1.2.9
	\label{nat:t:relprops}
	Let $a, b, c, d \in \mathbb{N}$.
	\begin{enumerate}
		\item \label{nat:t:relprops:1}
		      $a \leq a$ and $a \nless a$ and $a < a + 1$.
		\item \label{nat:t:relprops:2}
		      $1 \leq a$.
		\item \label{nat:t:relprops:3}
		      If $a < b$ and $b < c$, then $a < c$; if $a \leq b$ and $b < c$, then $a < c$; if $a < b$ and $b \leq c$, then $a < c$; if $a \leq b$ and $b \leq c$, then $a \leq c$.
		\item \label{nat:t:relprops:4}
		      $a < b$ if and only if $a + c < b + c$.
		\item \label{nat:t:relprops:5}
		      $a < b$ if and only if $a c < b c$.
		\item \xlabel[Trichotomy Law]{nat:t:relprops:trichotomy}
		      Precisely one of $a < b$ or $a = b$ or $a > b$ holds \quad (\nameref*{nat:t:relprops:trichotomy}).
		\item \label{nat:t:relprops:7}
		      $a \leq b$ or $b \leq a$.
		\item \label{nat:t:relprops:8}
		      If $a \leq b$ and $b \leq a$, then $a = b$.
		\item \label{nat:t:relprops:9}
		      It cannot be that $b < a < b + 1$.
		\item \label{nat:t:relprops:10}
		      $a \leq b$ if and only if $a < b + 1$.
		\item \label{nat:t:relprops:11}
		      $a < b$ if and only if $a + 1 \leq b$.
	\end{enumerate}
\end{theorem}

\begin{theorem}[Well-Ordering Principle] %% 1.2.10
	\label{nat:t:wop}
	Let $G \subseteq \mathbb{N}$ be a non-empty set. Then there is some $m \in G$ such that $m \leq g$ for all $g \in G$.
\end{theorem}


%-------------------------------------------------------------------------------------------------------------
\Newpage
\begin{exercise}[Used in Theorem \ref{nat:t:mult}] %% 1.2.1
	Fill in the missing details in the proof of Theorem \ref{nat:t:mult}.
\end{exercise}

\begin{proof}
	To prove uniqueness, suppose that there are two binary operations $\cdot: \mathbb{N} \times \mathbb{N} \to \mathbb{N}$ and $\odot: \mathbb{N} \times \mathbb{N} \to \mathbb{N}$ that satisfy the two properties of the theorem. Let
	\[
		G = \{ x \in \mathbb{N} \mid n \cdot x = n \odot x \text{ for all } n \in \mathbb{N}\}.
	\]
	We will prove that $G = \mathbb{N}$, which will imply that $\cdot$ and $\odot$ are the same operation. Clearly, $G \subseteq \mathbb{N}$. Also, $1 \in G$ because $n \cdot 1 = n = n \odot 1$ for all $n \in \mathbb{N}$. Now let $g \in G$, and let $n \in \mathbb{N}$. By hypothesis on $g$, we have $n \cdot g = n \odot g$. Then,
	\[
		n \cdot s(g) = (n \cdot g) + n = (n \odot g) + n = n \odot s(g).
	\]
	Therefore $s(g) \in G$, and hence by Part (\ref{nat:a:peano:induction}) of the \nameref{nat:a:peano} it follows that $G = \mathbb{N}$.

	\begin{notmine}
		Let $q \in \mathbb{N}$. Let $h_{q}: \mathbb{N} \to \mathbb{N}$ be defined by $h_{q}(m) = m + q$ for all $m \in \mathbb{N}$. Applying
		Theorem \ref{nat:t:recusion} to the set $\mathbb{N}$, the element $q \in \mathbb{N}$ and the function $h_{q} : \mathbb{N} \to \mathbb{N}$, implies that there is a unique function $g_{q}: \mathbb{N} \to \mathbb{N}$ such that $g_{q}(1) = q$ and $g_{q} \circ s = h_{q} \circ g_{q}$. Let $\cdot: \mathbb{N} \times \mathbb{N} \to \mathbb{N}$ be defined by $c \cdot d = g_{c}(d)$ for all $(c,d) \in \mathbb{N} \times \mathbb{N}$.
	\end{notmine}
	Then $n \cdot 1 = g_{n}(1) = n$, which is Part (\ref{nat:t:mult:1}), and
	\begin{align*}
		n \cdot s(m) & = g_{n}(s(m)) = (g_{n} \circ s)(m) = (h_{n} \circ g_{n})(m) \\
		             & = h_{n}(g_{n}(m)) = h_{n}(n \cdot m) = (n \cdot m) + n,
	\end{align*}
	which is Part (\ref{nat:t:mult:n}).
\end{proof}


%-------------------------------------------------------------------------------------------------------------
\Newpage
\begin{exercise}[Used in Theorem \ref{nat:t:props}] %% 1.2.2
	Prove Theorem \ref{nat:t:props} (\ref{nat:t:props:associative_add}) (\ref{nat:t:props:3}) (\ref{nat:t:props:commutative_add}) (\ref{nat:t:props:identity_mult}) (\ref{nat:t:props:distributive_right}) (\ref{nat:t:props:commutative_mult}) (\ref{nat:t:props:distributive_left}) (\ref{nat:t:props:associative_mult}) (\ref{nat:t:props:13}).
\end{exercise}

\begin{proof}
	\hfill

	\PartProof{nat:t:props:associative_add} Let
	\[
		G = \{ z \in \mathbb{N} \mid (x + y) + z = x + (y + z) \text{ for all } x, y \in \mathbb{N} \}.
	\]
	We will prove that $G = \mathbb{N}$, which will immediately imply the \nameref{nat:t:props:associative_add}. Clearly $G \subseteq \mathbb{N}$. Using both parts of Theorem \ref{nat:t:add} we have
	\[
		(a + b) + 1 = s(a + b) = a + s(b) = a + (b + 1).
	\]
	Hence, $1 \in G$. Now let $c \in G$. By hypothesis on $c$, we know that $(a + b) + c = a + (b + c)$, so by Part (\ref{nat:t:add:n}) of Theorem \ref{nat:t:add} it follows that
	\begin{align*}
		(a + b) + s(c) & = s((a + b) + c) = s(a + (b + c)) \\
		               & = a + s(b + c) = a + (b + s(c)).
	\end{align*}
	Therefore $s(c) \in G$, and hence $G = \mathbb{N}$ by Part (\ref{nat:a:peano:induction}) of the \nameref{nat:a:peano}.

	\PartProof{nat:t:props:3} Let
	\[
		G = \{ 1 + n = s(n) \text{ for all } n \in \mathbb{N} \}.
	\]
	We will prove that $G = \mathbb{N}$, which will immediately imply Part (\ref{nat:t:props:3}). Clearly $G \subseteq \mathbb{N}$, and $1 \in G$ because by Part (\ref{nat:t:add:1}) of Theorem \ref{nat:t:add} we have $1 + 1 = s(1)$. Now let $a \in G$. Using the \nameref{nat:t:props:associative_add} and Theorem \ref{nat:t:add}, it then follows that
	\begin{align*}
		1 + s(a) & = s(1 + a) = s(a + 1) = a + s(1)       \\
		         & = a + (1 + 1) = (a + 1) + 1 = s(a) + 1 \\
		         & = s(s(a)).
	\end{align*}
	Therefore $s(a) \in G$, and hence $G = \mathbb{N}$ by Part (\ref{nat:a:peano:induction}) of the \nameref{nat:a:peano}.

	\PartProof{nat:t:props:commutative_add} Let
	\[
		G = \{ x \in \mathbb{N} \mid x + y = y + x \text{ for all } y \in \mathbb{N} \}.
	\]
	We will prove that $G = \mathbb{N}$, which will immediately imply the \nameref{nat:t:props:commutative_add}. Clearly $G \subseteq \mathbb{N}$. By Part (\ref{nat:t:props:3}) we also have $1 + y = y + 1$ for all $y \in \mathbb{N}$, so $1 \in G$. Now let $a \in G$. Then,
	\begin{align*}
		s(a) + b & = (a + 1) + b                           & \text{(Part (a) of Theorem 1.2.5)}             \\
		         & = a + (1 + b)                           & \text{(\nameref{nat:t:props:associative_add})} \\
		         & = (1 + b) + a                           & \text{(Hypothesis on $a$)}                     \\
		         & = (b + 1) + a                           & \text{(Part (3) of the theorem)}               \\
		         & = b + (1 + a) = b + (a + 1) = b + s(a).
	\end{align*}
	Hence, $s(a) \in G$. We deduce that $G = \mathbb{N}$ by Part (\ref{nat:a:peano:induction}) of the \nameref{nat:a:peano}.

	\PartProof{nat:t:props:identity_mult} Let
	\[
		G = \{ 1 \cdot x = x \text{ for all x } \in \mathbb{N} \}.
	\]
	We will prove that $G = \mathbb{N}$, which will immediately imply the \nameref{nat:t:props:identity_mult}. Clearly $G \subseteq \mathbb{N}$. By Theorem \ref{nat:t:mult} (\ref{nat:t:mult:1}), it follows that $1 \cdot 1 = 1$, so $1 \in G$. Now let $a \in G$. Using Theorem \ref{nat:t:mult} (\ref{nat:t:mult:n}), hypothesis on $a$, and Theorem \ref{nat:t:add:1}, it then follows that
	\[
		1 \cdot s(a) = 1 \cdot a + 1 = a + 1 = s(a).
	\]
	Hence $s(a) \in G$, and by Part (\ref{nat:a:peano:induction}) of the \nameref{nat:a:peano}, we deduce that $G = \mathbb{N}$.

	\PartProof{nat:t:props:distributive_right} Let
	\[
		G = \{ z \in \mathbb{N} \mid (x + y)z = x z + y z \text{ for all } x, y \in \mathbb{N} \}.
	\]
	We will prove that $G = \mathbb{N}$, which will immediately imply the \nameref{nat:t:props:distributive_right}. Clearly $G \subseteq \mathbb{N}$. Also, by Theorem \ref{nat:t:mult} (\ref{nat:t:mult:n}), it follows that $(a + b) \cdot 1 = a + b = a \cdot 1 + b \cdot 1$, and as a result $1 \in G$. Now let $c \in G$. Then,
	\begin{align*}
		(a + b)s(c) & = (a + b)c + (a + b)    & \text{(Part (\ref{nat:t:mult:n}) of Theorem \ref{nat:t:mult})} \\
		            & = (a c + b c) + (a + b) & \text{(Hypothesis on $c$)}                                     \\
		            & = a c + (b c + (a + b)) & \text{(\nameref{nat:t:props:associative_add})}                 \\
		            & = a c + ((a + b) + b c) & \text{(\nameref{nat:t:props:commutative_add})}                 \\
		            & = (a c + (a + b)) + b c                                                                  \\
		            & = ((a c + a) + b) + b c                                                                  \\
		            & = (a c + a) + (b + b c)                                                                  \\
		            & = (a c + a) + (b c + b)                                                                  \\
		            & = a s(c) + b s(c)
	\end{align*}
	Hence, $s(c) \in G$, and by Part (\ref{nat:a:peano:induction}) of the \nameref{nat:a:peano}, we can conclude that $G \in \mathbb{N}$.

	\PartProof{nat:t:props:commutative_mult} Let
	\[
		G = \{ x \in \mathbb{N} \mid x y = y x \text{ for all } y \in \mathbb{N} \}.
	\]
	We will prove that $G = \mathbb{N}$, which will immediately imply the \nameref{nat:t:props:commutative_mult}. Clearly $G \subseteq \mathbb{N}$. By the \nameref{nat:t:props:identity_mult}, $1 \in G$. Now let $a \in G$. Then,
	\begin{align*}
		s(a) b & = (a + 1) b & \text{(Part (\ref{nat:t:add:1}) of Theorem \ref{nat:t:add})} \\
		       & = a b + b   & \text{(\nameref{nat:t:props:distributive_right})}            \\
		       & = b a + b   & \text{(Hypothesis on $a$)}                                   \\
		       & = b s(a).
	\end{align*}
	Hence, $s(a) \in G$, and we deduce that $G = \mathbb{N}$ because of Part (\ref{nat:a:peano:induction}) of the \nameref{nat:a:peano}.

	\PartProof{nat:t:props:distributive_left} Using Part (\ref{nat:t:props:distributive_right}) (\nameref{nat:t:props:distributive_right}) and Part (\ref{nat:t:props:commutative_mult}) (\nameref{nat:t:props:commutative_mult}), it follows that
	\[
		c(a + b) = (a + b)c = a c + b c = c a + c b.
	\]

	\PartProof{nat:t:props:associative_mult} Let
	\[
		G = \{ z \in \mathbb{N} \mid (x y)z = x(y z) \text{ for all } x, y \in \mathbb{N}  \}.
	\]
	We will prove that $G = \mathbb{N}$, which will immediately imply the \nameref{nat:t:props:associative_mult}. By Theorem \ref{nat:t:mult} (\ref{nat:t:mult:1}), it follows that $(a b) \cdot 1 = a b = a(b \cdot 1)$, and therefore $1 \in G$. Now let $a \in G$. Then,
	\begin{align*}
		(a b)s(c) & = (a b)c + a b & \text{(Part (b) of Theorem 1.2.6)}               \\
		          & = a(b c) + a b & \text{(Hypothesis on $c$)}                       \\
		          & = a(b c + b)   & \text{(\nameref{nat:t:props:distributive_left})} \\
		          & = a(b s(c)).
	\end{align*}
	Hence $s(c) \in G$. By Part (\ref{nat:a:peano:induction}) of the \nameref{nat:a:peano}, it then follows that $G = \mathbb{N}$.

	\PartProof{nat:t:props:13} Let $a b = 1$. Suppose to the contrary that either $a \not= 1$ or $b \not= 1$. Suppose, without loss of generality, that $a \not= 1$. By Lemma \ref{nat:l:exists_prev}, there is some $p \in \mathbb{N}$ such that $a = s(p)$, so $a b = s(p) b$. Because of the \nameref{nat:t:props:commutative_mult} and Theorem \ref{nat:t:mult} (\ref{nat:t:mult:n}), $s(p)b = b s(p) = b p + b = 1$, which is a contradiction to Part (\ref{nat:t:props:5}) of this theorem. Hence, $a = 1 = b$.

	Now let $a = 1 = b$. By Theorem \ref{nat:t:mult} (\ref{nat:t:mult:1}), it then follows that $a b = 1 \cdot 1 = 1$.
\end{proof}


%-------------------------------------------------------------------------------------------------------------
\Newpage
\begin{exercise}[Used in Section \ref{nat}] %% 1.2.3
	Let $a, b \in \mathbb{N}$. Suppose that $a < b$. Prove that there is a unique $p \in \mathbb{N}$ such that $a + p = b$.
\end{exercise}

\begin{proof}
	Since $a < b$, Definition \ref{nat:d:relation} immediately implies the existence part. For uniqueness, suppose that there are $p_{1}, p_{2} \in \mathbb{N}$ such that $a + p_{1} = b$ and $a + p_{2} = b$, so $a + p_{1} = a + p_{2}$. By the \nameref{nat:t:props:commutative_add}, it follows that $p_{1} + a = p_{2} + a$, and by the \nameref{nat:t:props:cancellation_add}, it follows that $p_{1} = p_{2}$.
\end{proof}


%-------------------------------------------------------------------------------------------------------------
\Newpage
\begin{exercise}[Used in Theorem \ref{nat:t:relprops}] %% 1.2.4
	Prove Theorem \ref{nat:t:relprops} (\ref{nat:t:relprops:1}) (\ref{nat:t:relprops:3}) (\ref{nat:t:relprops:4}) (\ref{nat:t:relprops:5}) (\ref{nat:t:relprops:11}).
\end{exercise}

\begin{proof}
	\hfill

	\PartProof{nat:t:relprops:1}
	As $a = a$, we have either $a < a$ or $a = a$, so $a \leq a$. Next suppose to the contrary that $a < a$. We can find some $p \in \mathbb{N}$ such that $a + p = a$, which is a contradiction to Theorem \ref{nat:t:relprops} (\ref{nat:t:relprops:trichotomy}). Hence, $a \nless a$. Finally, let $p = 1$. Then $a + p = a + 1$, and by Definition \ref{nat:d:relation} we deduce that $a < a + 1$.

	\PartProof{nat:t:relprops:3}
	Suppose that $a < b$ and $b < c$. We can choose some $p, q \in \mathbb{N}$ such that $a + p = b$ and $b + q = c$. Then $(a + p) + q = c$, and by the \nameref{nat:t:props:associative_add} we get $a + (p + q) = c$, which implies that $a < c$. Now suppose that $a \leq b$ and $b < c$. Then either $a < b$ or $a = b$. If $a < b$, then we already know that $a < c$. If $a = b$, then clearly $b = a < c$. A similar argument shows that if $a < b$ and $b \leq c$, then $a < c$. Finally, suppose that $a \leq b$ and $b \leq c$. This means either $a < b$ or $a = b$ and either $b < c$ or $b = c$. We have shown all cases above except the trivial case when we have $a = b = c$. Hence, either $a < c$ or $a = c$, or in other words $a \leq c$.

	\PartProof{nat:t:relprops:4}
	Suppose that $a < b$. We can find some $p \in \mathbb{N}$ such that $a + p = b$. Then $(a + p) + c = b + c$. Using Theorem \ref{nat:t:relprops} (\ref{nat:t:relprops:2}) (\ref{nat:t:relprops:4}), we have
	\begin{align*}
		(a + p) + c = a + (p + c) = a + (c + p) = (a + c) + p.
	\end{align*}
	Hence, $(a + c) + p = b + c$, which implies that $a + c < b + c$. It is possible to reverse these implications and deduce the argument.

	\PartProof{nat:t:relprops:5}
	Let
	\[
		G = \{ z \in \mathbb{N} \mid \text{ if } x < y \text{ then } x z < y z \}.
	\]
	We will prove that $G = \mathbb{N}$, which will imply this part of the theorem. Clearly $G \subseteq \mathbb{N}$. By Theorem \ref{nat:t:mult} (\ref{nat:t:mult:1}), it follows that if $a < b$ then $a = a \cdot 1 < b \cdot 1 = b$, and therefore $1 \in G$.  Now let $c \in G$. Suppose that $a < b$. By hypothesis on $c$, it then follows that $a c < b c$. Using the \nameref{nat:t:props:distributive_left}, Part (\ref{nat:t:relprops:4}), and the \nameref{nat:t:props:commutative_add}, we have
	\[
		a(c + 1) = a c + a < b c + a = a + b c < b + b c = b c + b = b(c + 1).
	\]
	Therefore $a(c + 1) < b(c + 1)$, and hence $c + 1 \in G$. By Part (\ref{nat:a:peano:induction}) of the \nameref{nat:a:peano}, it follows that $G = \mathbb{N}$, or in other words if $a < b$ then $a c < b c$.

	Now we will show that if $a c < b c$ then $a < b$. Suppose that $a c < b c$, and suppose to the contrary that $a \geq b$, which means either $a > b$ or $a = b$.
	\begin{bycases}
		\item $a = b$. Then $a c = b c$, which is a contradiction to the fact that $a c < b c$ because of the \nameref{nat:t:relprops:trichotomy}. Hence $a < b$.
		\item $a > b$. This means that $b < a$, and it then follows that $b c < a c$, again a contradiction. Hence $a < b$.
	\end{bycases}

	\PartProof{nat:t:relprops:11}
	Suppose that $a < b$, and suppose to the contrary that $b < a + 1$. By Part (\ref{nat:t:relprops:4}) it then follows that $a + 1 < b + 1$, so $b < a + 1 < b + 1$, which is a contradiction to Part (\ref{nat:t:relprops:9}) of this theorem. Hence, $a + 1 \leq b$.

	Now suppose that $a + 1 \leq b$, and suppose to the contrary that $b \leq a$. Then by Part (\ref{nat:t:relprops:3}) we deduce that $a + 1 \leq a$, which means either $a + 1 < a$ or $a + 1 = a$. If $a + 1 < a$ then there is a contradiction to Parts (\ref{nat:t:relprops:1}) and (\ref{nat:t:relprops:trichotomy}) of this theorem. If $a + 1 = a$ then there is a contradiction to Theorem \ref{nat:t:props} (\ref{nat:t:props:6}). Hence, $a < b$.
\end{proof}


%-------------------------------------------------------------------------------------------------------------
\Newpage
\begin{exercise}[Used in Exercise \ref{int:e:3}] %% 1.2.5
	\label{nat:e:5}
	Let $a, b \in \mathbb{N}$. Prove that if $a + a = b + b$, then $a = b$.
\end{exercise}

\begin{proof}
	Suppose that $a + a = b + b$. Using the \nameref{nat:t:props:distributive_left} and Theorem \ref{nat:t:mult} (\ref{nat:t:mult:n}), we have
	\[
		a(1 + 1) = a \cdot 1 + a \cdot 1 = a + a = b + b = b \cdot 1 + b \cdot 1 = b(1 + 1).
	\]
	Thus, $a(1 + 1) = b(1 + 1)$. Then by the \nameref{nat:t:props:cancellation_mult} we deduce that $a = b$.
\end{proof}


%-------------------------------------------------------------------------------------------------------------
\Newpage
\begin{exercise} %% 1.2.6
	Let $b \in \mathbb{N}$. Prove that
	\begin{align*}
		\{n \in \mathbb{N} \mid 1 \leq n \leq b\} \cup\{n \in \mathbb{N} \mid b+1 \leq n\} = \mathbb{N} \\
		\{n \in \mathbb{N} \mid 1 \leq n \leq b\} \cap\{n \in \mathbb{N} \mid b+1 \leq n\} = \emptyset.
	\end{align*}
\end{exercise}

\begin{proof}
	Let
	\[
		G = \{n \in \mathbb{N} \mid 1 \leq n \leq b\} \cup\{n \in \mathbb{N} \mid b+1 \leq n\}.
	\]
	We will prove that $G = \mathbb{N}$. Clearly $G \subseteq \mathbb{N}$. By Theorem \ref{nat:t:relprops} (\ref{nat:t:relprops:1}) (\ref{nat:t:relprops:2}) it follows that $1 \leq 1 \leq b$. Hence, $1 \in G$. Now let $a \in G$. We consider the following three cases:
	\begin{bycases}
		\item $a < b$. By Theorem \ref{nat:t:relprops} (\ref{nat:t:relprops:4}) it then follows that $a + 1 < b + 1$, and by Definition \ref{nat:d:relation} it follows that $a + 1 \leq b$. Hence $a + 1 \in G$.
		\item $a = b$. Then $a + 1 = b + 1$, so $a + 1 \leq b + 1$, and hence $a + 1 \in G$.
		\item $a > b$. By Theorem \ref{nat:t:relprops} (\ref{nat:t:relprops:4}) we get $b + 1 \leq a + 1$. Hence $a + 1 \in G$.
	\end{bycases}
	Thus, $a + 1 \in G$, and by Part (\ref{nat:a:peano:induction}) of the \nameref{nat:a:peano} we deduce that $G = \mathbb{N}$.

	Now let
	\[
		M = \{n \in \mathbb{N} \mid 1 \leq n \leq b\} \cap\{n \in \mathbb{N} \mid b+1 \leq n\}.
	\]
	Suppose to the contrary that $M \not= \emptyset$. We can choose some $a \in M$ such that $1 \leq a \leq b$ and $b + 1 \leq a$. Using Theorem \ref{nat:t:relprops} (\ref{nat:t:relprops:10}), we have $a < b + 1$ and $b + 1 < a + 1$, or in other words $a < b + 1 < a + 1$, which is a contradiction to Theorem \ref{nat:t:relprops} (\ref{nat:t:relprops:9}). Therefore $M = \emptyset$.
\end{proof}


%-------------------------------------------------------------------------------------------------------------
\Newpage
\begin{exercise} %% 1.2.7
	Let $A \subseteq \mathbb{N}$ be a set. The set $A$ is \emph{closed} if $a \in A$ implies $a + 1 \in A$. Suppose that $A$ is closed.
	\begin{enumerate}
		\item \label{nat:e:7:1}
		      Prove that if $a \in A$ and $n \in \mathbb{N}$, then $a +n  \in A$.
		\item \label{nat:e:7:2}
		      Prove that if $a \in A$, then $\{x \in \mathbb{N} \mid x \geq a\} \subseteq A$.
	\end{enumerate}
\end{exercise}

\begin{proof}[(\ref{nat:e:7:1})]
	Let
	\[
		G = \{ n \in \mathbb{N} \mid \text{ if } a \in A \text{ then } a + n \in A \}.
	\]
	We will prove that $G = \mathbb{N}$, which will imply the desired result. Clearly $G \subseteq \mathbb{N}$. Because $A$ is closed, we deduce that $1 \in G$. Now let $a \in A$ and $n \in G$. Then $a + n \in A$, and since $A$ is closed, it follows that $(a + n) + 1 \in A$. Using the \nameref{nat:t:props:associative_add}, we obtain $(a + n) + 1 = a + (n + 1)$, and hence $n + 1 \in G$. Thus, by Part (\ref{nat:a:peano:induction}) of the \nameref{nat:a:peano}, we can conclude that $G = \mathbb{N}$.
\end{proof}

\begin{proof}[(\ref{nat:e:7:2})]
	Let $a \in A$, and suppose that $n \in \{ x \in \mathbb{N} \mid x \geq a \}$. Then $a \leq n$, which means either $a < n$ or $a = n$. If $a = n$ then clearly $n \in A$. Now suppose that $a \not= n$, so $a < n$. According to Definition \ref{nat:d:relation} there is some $k \in \mathbb{N}$ such that $a + k = n$. By Part (\ref{nat:e:7:1}) of this exercise, it follows that $a + k \in A$, or in other words $n \in A$. Since $n$ was arbitrary, we can conclude that $\{ x \in \mathbb{N} \mid x \geq a \} \subseteq A$.
\end{proof}


%-------------------------------------------------------------------------------------------------------------
\Newpage
\begin{exercise}[Used in Section \ref{nat}] %% 1.2.8
	Suppose that the set $\mathbb{N}$ together with the element $1 \in \mathbb{N}$ and the function $s: \mathbb{N} \to \mathbb{N}$, and that the set $\mathbb{N}'$ together with the element $1' \in \mathbb{N}'$ and the function $s': \mathbb{N}' \to \mathbb{N}'$, both satisfy the \nameref{nat:a:peano}. Prove that there is a bijective function $f: \mathbb{N} \to \mathbb{N}^{\prime}$ such that $f(1)=1'$ and $f \circ s=s' \circ f$. The existence of such a bijective function proves that the natural numbers are essentially unique.

	The existence of the function $f$ follows immediately from the existence part of Theorem \ref{nat:t:recusion}; the trickier aspect of this exercise is to prove that $f$ is bijective. To do that, find an inverse for $f$ by using the existence part of Theorem \ref{nat:t:recusion} again, and then prove that the function you found is an inverse of $f$ by using the uniqueness part of Theorem \ref{nat:t:recusion}.
\end{exercise}

\begin{proof}
	We will prove that $f$ is a bijective function. Because the set $\mathbb{N}'$ together with the element $1' \in \mathbb{N}'$ and the function $s': \mathbb{N}' \to \mathbb{N}'$ satisfies the \nameref{nat:a:peano}, we can imply Theorem \ref{nat:t:recusion} to the set $\mathbb{N}$, the element $1 \in \mathbb{N}$ and the function $s: \mathbb{N} \to \mathbb{N}$, to deduce that there is a unique function $g: \mathbb{N} \to \mathbb{N}'$ such that $g(1') = 1$ and $g \circ s'$ = $s \circ g$.

	Let
	\[
		G = \{ n \in \mathbb{N} \mid g(f(n)) = n \}.
	\]
	We will show that $G = \mathbb{N}$, which will imply that $g(f(n)) = n$ for all $n \in \mathbb{N}$. Since ${g(f(1)) = g(1') = 1}$, it follows that $1 \in G$. Now let $n \in G$, which means that $g(f(n)) = n$. Then,
	\begin{align*}
		g(f(s(n))) & = (g \circ f \circ s)(n) = (g \circ (f \circ s))(n) = (g \circ (s' \circ f))(n) \\
		           & = (g \circ s' \circ f)(n) = (s \circ g \circ f)(n) = s(g(f(n)))                 \\
		           & = s(n).
	\end{align*}
	Hence, $s(n) \in G$, and by Part (\ref{nat:a:peano:induction}) of the \nameref{nat:a:peano}, it follows that $G = \mathbb{N}$. A similar argument shows that $f(g(n')) = n'$ for all $n' \in \mathbb{N}'$.

	Now let $n \in \mathbb{N}$ and $n' \in \mathbb{N}'$. If $n' = f(n)$ then $g(n') = g(f(n)) = n$. Also, if $n = g(n')$ then $f(n) = f(g(n')) = n'$. Because of uniqueness of $g$, it then follows that $g$ is an inverse of $f$, and we can conclude that $f$ is bijective.
\end{proof}


%-------------------------------------------------------------------------------------------------------------
\Newpage
\begin{exercise}[Not in the book] %% 1.2.9
	Given any two of the three axioms of the \nameref{nat:a:peano}, find a structure that satisfies those two axioms, but not the third. Feel free to assume $\mathbb{R}$, $\mathbb{Z}$, or anything else for this problem.
\end{exercise}

\subsection*{Example 1}
Suppose that $n \in \mathbb{N}$. Let $A = \{ x \in \mathbb{N} \mid x \leq n \}$, and let $s: A \to A$ be a function that is defined by
\[
	s(x) = \begin{cases}
		x + 1 & x < n            \\
		1     & \text{otherwise}
	\end{cases}.
\]
Then they satisfy the following two properties.
\begin{lenumerate}
	\item The function $s$ is injective.
	\item Let $G \subseteq A$ be a set. Suppose that $1 \in G$, and that if $g \in G$ then $s(g) \in G$. Then
	      $G = A$.
\end{lenumerate}
But then $s(n) = 1$.

\subsection*{Example 2}
Suppose that $n \in \mathbb{N}$. Let $A = \{ x \in \mathbb{N} \mid x \leq n \}$, and let $s: A \to A$ be a function that is defined by
\[
	s(x) = \begin{cases}
		x + 1 & x < n            \\
		x     & \text{otherwise}
	\end{cases}.
\]
Then they satisfy the following two properties.
\begin{lenumerate}
	\item There is no $n \in A$ such that $s(n) = 1$.
	\item Let $G \subseteq A$ be a set. Suppose that $1 \in G$, and that if $g \in G$ then $s(g) \in G$. Then
	      $G = A$.
\end{lenumerate}
We can find some $m \in A$ such that $m + 1 = n$. Then $s(m) = n = s(n)$ and $m \not= n$, and hence $s$ is not injective.

\subsection*{Example 3}
Let $A = \mathbb{N}$, and let $s: A \to A$ be a function that is defined by $s(x) = x + 2$. Then they satisfy the following two properties.
\begin{lenumerate}
	\item There is no $n \in A$ such that $s(n) = 1$.
	\item The function $s$ is injective.
\end{lenumerate}
Let $G = \{ a \in A \mid  a \text{ is even } \}$. Clearly $G \subseteq A$ and $1 \in G$. Also, if $g \in G$ then $s(g) \in G$. But it is obvious that $G \not= A$.


\Newpage
%-------------------------------------------------------------------------------------------------------------
\begin{exercise}[Not in the book] %% 1.2.10
	Construct the exponentiation $n^m$ and prove its basic properties.
\end{exercise}

\begin{theorem}[Exponentiation]
	\label{nat:t:exp}
	There is a unique binary operation $\boxdot: \mathbb{N} \times \mathbb{N} \to \mathbb{N}$ that satisfies the following two properties for all $n, m \in \mathbb{N}$.
	\begin{lenumerate}
		\item \label{nat:t:exp:1}
		      $n \boxdot 1 = n$.
		\item \label{nat:t:exp:n}
		      $n \boxdot s(m) = (n \boxdot m) \cdot n$.
	\end{lenumerate}
	The number $n \boxdot m$ is also denoted $n^m$.
\end{theorem}

\begin{proof}
	To prove uniqueness, suppose that there are two binary operations $\boxdot: \mathbb{N} \times \mathbb{N} \to \mathbb{N}$ and $\boxtimes: \mathbb{N} \times \mathbb{N} \to \mathbb{N}$ that satisfy the two properties of this theorem. Let
	\[
		G = \{ g \in \mathbb{N} \mid n \boxdot g = n \boxtimes g \text{ for all } n \in \mathbb{N} \}.
	\]
	We will prove that $G = \mathbb{N}$, which will imply that $\boxdot$ and $\boxtimes$ are the same opertion. Clearly $G \subseteq \mathbb{N}$. Because $n \boxdot 1 = n = n \boxtimes 1$ it follows that $1 \in G$. Now let $n \in \mathbb{N}$ and $g \in G$. This means that $n \boxdot g = n \boxtimes g$. Then,
	\[
		n \boxdot s(g) = (n \boxdot g) \cdot n = (n \boxtimes g) \cdot n = n \boxtimes s(g).
	\]
	Hence $s(g) \in G$, and by Part (\ref{nat:a:peano:induction}) of the \nameref{nat:a:peano} we can conclude that $G = \mathbb{N}$.

	For existence, let $p \in \mathbb{N}$, and let $k_{p}: \mathbb{N} \to \mathbb{N}$ be defined by $k_{p}(n) = p \cdot n$ for all $n \in \mathbb{N}$. We can apply Theorem \ref{nat:t:recusion} to the set $\mathbb{N}$, the element $p \in \mathbb{N}$ and the function $k_{p}: \mathbb{N} \to \mathbb{N}$, to deduce that there is a unique function $f_{p}: \mathbb{N} \to \mathbb{N}$ such that $f_{p}(1) = p$ and $f_{p} \circ s = k_{p} \circ f_{p}$. Let $\boxdot: \mathbb{N} \to \mathbb{N}$ be defined by $c \boxdot d = f_{c}(d)$ for all $(c, d) \in \mathbb{N}$. Let $n, m \in \mathbb{N}$. It then follows that $n \boxdot 1 = f_{n}(1) = n$, which is Part (\ref{nat:t:exp:1}), and also we have
	\begin{align*}
		n \boxdot s(m) & = f_{n}(s(m)) = (f_{n} \circ s)(m) = (k_{n} \circ f_{n})(m)     \\
		               & = k_{n}(f_{n}(m)) = k_{n}(n \boxdot m) = (n \boxdot m) \cdot n,
	\end{align*}
	which is Part (\ref{nat:t:exp:n}).
\end{proof}

\begin{theorem}
	\label{nat:t:exp_props}
	Let $a, b, c \in \mathbb{N}$.
	\begin{enumerate}
		\item \label{nat:t:exp_props:1}
		      $a^{b + c} = a^b a^c$.
		\item \label{nat:t:exp_props:2}
		      $a^{b c} = (a^b)^c$.
		\item \label{nat:t:exp_props:3}
		      $b^a c^a = (b c)^a$.
	\end{enumerate}
\end{theorem}

\begin{proof}
	\hfill

	\PartProof{nat:t:exp_props:1}
	Let
	\[
		G = \{ y \in \mathbb{N} \mid  a^{x + y} = a^x a^y \text{ for all } x \in \mathbb{N} \}.
	\]
	We will prove that $G = \mathbb{N}$, which will immediately imply that $a^{b + c} = a^b a^c$. Clearly $G \in \mathbb{N}$. By Theorem \ref{nat:t:exp} it follows that
	\[
		a^{b + 1} = a^{s(b)} = a^b a = a^b a^1,
	\]
	and hence $1 \in G$. Now let $c \in G$, or in other words $a^{b + c} = a^b a^c$. By repeated use of the \hyperref[nat:t:props:associative_add]{Associative} and \hyperref[nat:t:props:commutative_add]{Commutative} Laws for Addition and \nameref{nat:t:props:associative_mult} we obtain
	\begin{align*}
		a^{b + s(c)} & = a^{b + (c + 1)} = a^{b + (1 + c)} = a^{(b + 1) + c} = a^{s(b + c)} \\
		             & = a^{b + c} a  = (a^b a^c)a = a^b (a^c a) = a^b a^{s(c)},
	\end{align*}
	and hence $s(c) \in G$. Thus, by Part (\ref{nat:a:peano:induction}) of the \nameref{nat:a:peano} we can conclude that $G = \mathbb{N}$.

	\PartProof{nat:t:exp_props:2}
	Let
	\[
		G = \{ y \in \mathbb{N} \mid  a^{x y} = (a^x)^y \text{ for all } x \in \mathbb{N} \}.
	\]
	We will prove that $G = \mathbb{N}$, which will immediately imply that $a^{b c} = (a^b)^c$. Clearly $G \in \mathbb{N}$. By Part (\ref{nat:t:exp:1}) of Theorem \ref{nat:t:exp} we deduce that $a^{b \cdot 1} = a^b = (a^b)^1$, and hence $1 \in G$. Now let $c \in G$. This means that $a^{b c} = (a^b)^c$. Using the \nameref{nat:t:props:distributive_left} we obtain
	\[
		a^{b \cdot s(c)} = a^{b(c + 1)} = a^{b c + b} = a^{b c} a^b = (a^b)^c a^b = (a^b)^{s(c)}.
	\]
	Therefore $s(c) \in G$, and hence by Part (\ref{nat:a:peano:induction}) of the \nameref{nat:a:peano} we can conclude that $G = \mathbb{N}$.

	\PartProof{nat:t:exp_props:3}
	Let $G = \{ x \in \mathbb{N} \mid  b^x c^x = (b c)^x \}$. We will prove that $G = \mathbb{N}$, which will immediately imply that $b^a c^a = (b c)^a$. Clearly $G \in \mathbb{N}$. Because of Part (\ref{nat:t:exp:1}) of Theorem\,\ref{nat:t:exp} it follows that $b^1 c^1 = b c = (b c)^1$. Now let $a \in G$. Then $b^a c^a = (b c)^a$. Using Part (\ref{nat:t:exp:n}) of Theorem \ref{nat:t:exp} we deduce that $b^{s(a)} = b^a b$ and $c^{s(a)} = c^a c$. By repeated use of the \hyperref[nat:t:props:associative_mult]{Associative} and \hyperref[nat:t:props:commutative_mult]{Commutative} Laws for Multiplication we obtain
	\begin{align*}
		b^{s(a)} c^{s(a)} & = (b^a b)(c^a c) = b^a (b c^a) c = b^a (c^a b) c \\
		                  & = (b^a c^a)(b c) = (b c)^a (b c) = (b c)^{s(a)}.
	\end{align*}
	Hence $s(a) \in G$, and because of Part (\ref{nat:a:peano:induction}) of the \nameref{nat:a:peano} we can conclude that $G = \mathbb{N}$.
\end{proof}