%=============================================================================================================
%= SECTION 1.4
%=============================================================================================================
\section{Entry 2: Axioms for the Integers}
\label{int2}

\begin{definition} %% 1.4.1
	\xlabel[ordered integral domain]{int2:d:oid}
	An \emph{\nameref*{int2:d:oid}} is a set $R$ with elements $0,1 \in R$, binary operations $+$ and $\cdot$, a unary operation $-$ and a relation $<$, which satisfy the following properties. Let $x, y, z \in R$.
	\begin{lenumerate}
		\item \xlabel[Associative Law for Addition]{int2:d:oid:associative_add}
		      $(x + y) + z = x + (y + z)$ \quad (\nameref*{int2:d:oid:associative_add}).
		\item \xlabel[Commutative Law for Addition]{int2:d:oid:commutative_add}
		      $x + y = y + x$ \quad (\nameref*{int2:d:oid:commutative_add}).
		\item \xlabel[Identity Law for Addition]{int2:d:oid:identity_add}
		      $x + 0 = x$ \quad (\nameref*{int2:d:oid:identity_add}).
		\item \xlabel[Inverses Law for Addition]{int2:d:oid:inverses_add}
		      $x + (-x) = 0$ \quad (\nameref*{int2:d:oid:inverses_add}).
		\item \xlabel[Associative Law for Multiplication]{int2:d:oid:associative_mult}
		      $(x y) z = x (y z)$ \quad (\nameref*{int2:d:oid:associative_mult}).
		\item \xlabel[Commutative Law for Multiplication]{int2:d:oid:commutative_mult}
		      $x y = y x$ \quad (\nameref*{int2:d:oid:commutative_mult}).
		\item \xlabel[Identity Law for Multiplication]{int2:d:oid:identity_mult}
		      $x \cdot 1 = x$ \quad (\nameref*{int2:d:oid:identity_mult}).
		\item \xlabel[Distributive Law]{int2:d:oid:distributive}
		      $x (y + z) = x y + x z$ \quad (\nameref*{int2:d:oid:distributive}).
		\item \xlabel[No Zero Divisors Law]{int2:d:oid:no_zero_divisors}
		      If $x y = 0$, then $x = 0$ or $y = 0$ \quad (\nameref*{int2:d:oid:no_zero_divisors}).
		\item \xlabel[Trichotomy Law]{int2:d:oid:trichotomy}
		      Precisely one of $x < y$ or $x = y$ or $x > y$ holds \quad (\nameref*{int2:d:oid:trichotomy}).
		\item \xlabel[Transitive Law]{int2:d:oid:transitive}
		      If $x < y$ and $y < z$, then $x < z$ \quad (\nameref*{int2:d:oid:transitive}).
		\item \xlabel[Addition Law for Order]{int2:d:oid:addition_order}
		      If $x < y$ then $x + z < y + z$ \quad (\nameref*{int2:d:oid:addition_order}).
		\item \xlabel[Multiplication Law for Order]{int2:d:oid:multiplication_order}
		      If $x < y$ and $z > 0$, then $x z < y z$ \quad (\nameref*{int2:d:oid:multiplication_order}).
		\item \xlabel[Non-Triviality]{int2:d:oid:non_triviality}
		      $0 \neq 1$ \quad (\nameref*{int2:d:oid:non_triviality}).
	\end{lenumerate}
\end{definition}

\begin{definition} %% 1.4.2
	Let $R$ be an \nameref{int2:d:oid}, and let $A \subseteq R$ be a set.
	\begin{enumerate}
		\item The relation $\leq$ on $R$ is defined by $a \leq b$ if and only if $a < b$ or $a = b$, for all $a, b \in R$.
		\item \xlabel[least element]{int2:d:least_element}
		      The set $A$ has a \emph{\nameref*{int2:d:least_element}} if there is some $a \in A$ such that $a \leq x$ for all $x \in A$.
	\end{enumerate}
\end{definition}

\begin{definition} %% 1.4.3
	\xlabel[Well-Ordering Principle]{int2:d:wop}
	Let $R$ be an \nameref{int2:d:oid}. The ordered integral domain $R$ satisfies the \emph{\nameref*{int2:d:wop}} if every non-empty subset of $\{x \in R \mid x > 0 \}$
	has a \nameref{int2:d:least_element}.
\end{definition}

\begin{axiom}[Axiom for the Integers]  %% 1.4.4
	There exists an \nameref{int2:d:oid} $\mathbb{Z}$ that satisfies the \nameref{int2:d:wop}.
\end{axiom}

\begin{lemma} %% 1.4.5
	\label{int2:l:props}
	Let $x, y, z \in \mathbb{Z}$.
	\begin{enumerate}
		\item \xlabel[Cancellation Law for Addition]{int2:l:props:cancellation_add}
		      If $x + z = y + z$, then $x = y$ \quad (\nameref*{int2:l:props:cancellation_add}).
		\item \label{int2:l:props:2}
		      $-(-x) = x$.
		\item \label{int2:l:props:3}
		      $-(x + y) = (-x) + (-y)$.
		\item \label{int2:l:props:4}
		      $x \cdot 0 = 0$.
		\item \xlabel[Cancellation Law for Multiplication]{int2:l:props:cancellation_mult}
		      If $z \neq 0$ and if $x z = y z$, then $x = y$ \quad (\nameref*{int2:l:props:cancellation_mult}).
		\item \label{int2:l:props:6}
		      $(-x) y = -x y = x(-y)$.
		\item \label{int2:l:props:7}
		      $x y = 1$ if and only if $x = 1 = y$ or $x = -1 = y$.
		\item \label{int2:l:props:8}
		      $x > 0$ if and only if $-x < 0$, and $x < 0$ if and only if $-x > 0$.
		\item \label{int2:l:props:9}
		      $0 < 1$.
		\item \label{int2:l:props:10}
		      If $x \leq y$ and $y \leq x$, then $x = y$.
		\item \label{int2:l:props:11}
		      If $x > 0$ and $y > 0$, then $x y > 0 .$ If $x > 0$ and $y < 0$, then $x y < 0$.
	\end{enumerate}
\end{lemma}

\begin{theorem} %% 1.4.6
	\label{int2:t:discrete}
	Let $x \in \mathbb{Z}$. Then there is no $y \in \mathbb{Z}$ such that $x < y < x + 1$.
\end{theorem}

\begin{definition} %% 1.4.7
	\hfill
	\begin{enumerate}
		\item Let $x \in \mathbb{Z}$. The number $x$ is positive if $x > 0$, and the number $x$ is negative if $x < 0$.
		\item The set of natural numbers, denoted $\mathbb{N}$, is defined by
		      $$
			      \mathbb{N} = \{ x \in \mathbb{Z} \mid x > 0 \}.
		      $$
	\end{enumerate}
\end{definition}

\begin{theorem}[Peano Postulates] %% 1.4.8
	\label{int2:t:peano}
	Let $s: \mathbb{N} \to \mathbb{N}$ be defined by $s(n) = n + 1$ for all $n \in \mathbb{N}$.
	\begin{lenumerate}
		\item There is no $n \in \mathbb{N}$ such that $s(n) = 1$.
		\item The function $s$ is injective.
		\item Let $G \subseteq \mathbb{N}$ be a set. Suppose that $1 \in G$, and that if $g \in G$ then $s(g) \in G$. Then
		      $G = \mathbb{N}$. \label{int2:t:peano:induction}
	\end{lenumerate}
\end{theorem}


%-------------------------------------------------------------------------------------------------------------
\Newpage
\begin{restatable}[Not in the book]{lemma}{intiiLessRelL} %% 1.4.9
	\label{int2:l:less_relation}
	Let $a, b \in \mathbb{Z}$. Suppose that $a < b$. Then there is some $k \in \mathbb{N}$ such that $a + k = b$.
\end{restatable}

\begin{proof}
	Let $k = b + (-a)$. By the \nameref{int2:d:oid:addition_order} we have $b + (-a) > a + (-a)$, and by the \nameref{int2:d:oid:inverses_add} we deduce that $k = b + (-a) > 0$. Hence, $k \in \mathbb{N}$. Now adding $a$ to both sides of the equation $k = b + (-a)$ we obtain $k + a = b + (-a) + a$. Using the \hyperref[int2:d:oid:associative_add]{Associative} and \hyperref[int2:d:oid:commutative_add]{Commutative} Laws for Addition it follows that $a + k = b + [a + (-a)]$, and hence by the \hyperref[int2:d:oid:inverses_add]{Inverses} and \hyperref[int2:d:oid:identity_add]{Identity} Laws for Addition we see that $a + k = b$.
\end{proof}


\addtocounter{exercise}{1}
%-------------------------------------------------------------------------------------------------------------
\Newpage
\begin{exercise}[Used in Section \ref{int2}] %% 1.4.2
	Let $n \in \mathbb{N}$. Prove that $n + 1 \in \mathbb{N}$.
\end{exercise}

\begin{proof}
	Suppose that $n \in \mathbb{N}$. Then $n \in \mathbb{Z}$ and $n > 0$. Using the \nameref{int2:d:oid:addition_order} we obtain $n + 1 > 0 + 1$. By the \hyperref[int2:d:oid:commutative_add]{Commutative} and \hyperref[int2:d:oid:identity_add]{Identity} Laws for Addition we deduce that $n + 1 > 1$. Also, we know by Part (\ref{int2:l:props:9}) of Lemma\,\ref{int2:l:props} that $0 < 1$. Then applying the \nameref{int2:d:oid:transitive} we get $n + 1 > 0$, and hence by the definition of $\mathbb{N}$ we see that $n + 1 \in \mathbb{N}$.
\end{proof}


%-------------------------------------------------------------------------------------------------------------
\Newpage
\begin{exercise}[Used in Exercise \ref{int2:e:8}] %% 1.4.3
	\label{int2:e:3}
	Let $x, y \in \mathbb{Z}$. Prove that $x \leq y$ if and only if $-x \geq -y$.
\end{exercise}

\begin{proof}
	Suppose that $x \leq y$, which means that either $x = y$ or $x < y$. Suppose that $x = y$. Then $x(-1) = y(-1)$, and by Part (\ref{int2:l:props:6}) of Lemma \ref{int2:l:props} we obtain
	$-(x \cdot 1) = -(y \cdot 1)$. Hence by the \nameref{int2:d:oid:identity_mult} we see that $-x = -y$.

	Now suppose that $x < y$, or in other words $y > x$. Then by the \nameref{int2:d:oid:addition_order} we deduce that $y + (-x) + (-y) > x + (-x) + (-y)$. Using the \hyperref[int2:d:oid:commutative_add]{Commutative} and \hyperref[int2:d:oid:associative_add]{Associative} Laws for Addition we get $(-x) + [y + (-y)] > (-y) + [x + (-x)]$. But then by the \hyperref[int2:d:oid:inverses_add]{Inverses} and \hyperref[int2:d:oid:identity_add]{Identity} Laws for Addition we can conclude that $-x > -y$.

	Thus, $-x \geq -y$. This process can be done backwards, and hence $x \leq y$ if and only if $-x \geq -y$.
\end{proof}


%-------------------------------------------------------------------------------------------------------------
\Newpage
\begin{exercise}[Used in Exercise \ref{int2:e:6}, Exercise \ref{int2:e:8} and Exercise \ref{rat:e:9}] %% 1.4.4
	\label{int2:e:4}
	Prove that $\mathbb{N} = \{ x \in \mathbb{Z} \mid x \geq 1 \}$.
\end{exercise}

\begin{proof}
	We will prove that $x > 0$ if and only if $x \geq 1$ for all $x \in \mathbb{Z}$, which will imply that $\mathbb{N} = \{ x \in \mathbb{Z} \mid x \geq 1 \}$. Let $x \in \mathbb{Z}$, and suppose that $x > 0$. Then by the \nameref{int2:d:wop} it follows that $x \geq 0 + 1$, and hence by the \hyperref[int2:d:oid:commutative_add]{Commutative} and \hyperref[int2:d:oid:identity_add]{Identity} Laws for Addition we can conclude that $x \geq 1$. Now suppose that $x \geq 1$, which means that either $x = 1$ or $x > 1$. Because of Lemma \ref{int2:l:props} (\ref{int2:l:props:9}) we know that $0 < 1$, so if $x = 1$ then clearly $x > 0$. But if $x > 1$ then $0 < 1 < x$, and by the \nameref{int2:d:oid:transitive} we deduce that $x > 0$.

\end{proof}


%-------------------------------------------------------------------------------------------------------------
\Newpage
\begin{exercise} %% 1.4.5
	Let $a, b \in \mathbb{Z}$. Prove that if $a < b$, then $a + 1 \leq b$.
\end{exercise}

\begin{proof}
	If $a < b$ then by the \nameref{int2:d:wop} it follows that $a + 1 \leq b$.
\end{proof}


%-------------------------------------------------------------------------------------------------------------
\Newpage
\begin{exercise}[Used in Theorem \ref{irp:t:ind}] %% 1.4.6
	\label{int2:e:6}
	Let $n \in \mathbb{N}$. Suppose that $n \neq 1$. Prove that there is some $b \in \mathbb{N}$ such that $b + 1 = n$.
	\hfill [Use Exercise \ref{int2:e:4}.]
\end{exercise}

\begin{proof}
	Let $b = n + (-1)$. By Exercise \ref{int2:e:4} we know that $n \geq 1$, and because $n \not= 1$ it implies that $n > 1$. By the \nameref{int2:d:oid:addition_order} it follows that $n + (-1) > 1 + (-1)$, and by the \nameref{int2:d:oid:inverses_add} we deduce that $n + (-1) > 0$, so $b > 0$. Hence, $b \in \mathbb{N}$.

	We have $b + 1 = n + (-1) + 1$. By the \hyperref[int2:d:oid:commutative_add]{Commutative} and \hyperref[int2:d:oid:inverses_add]{Inverses} Laws for Addition we see that $(-1) + 1 = 1 + (-1) = 0$. Then applying the \nameref{int2:d:oid:associative_add} we deduce that $n + [(-1) + 1] = n + 0$, and hence by the \nameref{int2:d:oid:identity_add} we can conclude that $b + 1 = n$.
\end{proof}


\Newpage
%-------------------------------------------------------------------------------------------------------------
\begin{exercise}[Used in Section \ref{int2}] %% 1.4.7
	Let $\mathbb{Z}[x]$ denote the set of polynomials with integer coefficients and variable $x$. This set has binary operations $+$ and $\cdot$ as usual for polynomials. The relation $<$, called the \emph{dictionary order} on $\mathbb{Z}[x]$, is defined by $f < g$ if and only if either the degree of $f$ is less than the degree of $g$, or if the degrees of $f$ and $g$ are equal and if $f \neq g$ and if the highest degree coefficient which differs for $f$ and $g$ is smaller for $f$, for all $f, g \in \mathbb{Z}[x]$. Let $0, 1 \in \mathbb{Z}[x]$ be the polynomials that are constantly $0$ and $1$, respectively.
	\begin{enumerate}
		\item \label{int2:e:7:1}
		      Prove that $\mathbb{Z}[x]$, with $+$, $\cdot$, $<$, $0$ and $1$ as defined above, is an \nameref{int2:d:oid}.
		\item \label{int2:e:7:2}
		      Let $f \in \mathbb{Z}[x]$. Prove that there is no $g \in \mathbb{Z}[x]$ such that $f < g < f + 1$.
		\item \label{int2:e:7:3}
		      Prove that $\mathbb{Z}[x]$ does not satisfy the \nameref{int2:d:wop}.
	\end{enumerate}
\end{exercise}

\begin{definition}
	The set of \emph{polynomials with variable $x$}, denoted $\mathbb{Z}[x]$, is the set of infinite sequences with integer coefficients in ascending order of degrees.

	The elements $\widetilde{0}, \widetilde{1} \in \mathbb{Z}[x]$ are defined by $\widetilde{0} = (0, 0, 0, \ldots)$ and $\widetilde{1} = (1, 0,0, \ldots)$. The binary operations $+$ and $\cdot$ on $\mathbb{Z}[x]$ are defined by
	\begin{align*}
		(a_0, a_1, a_2, \ldots) + (b_0, b_1, b_2, \ldots)     & = (a_0 + b_0,\; a_1 + b_1,\; a_2 + b_2,\; \ldots) \\
		(a_0, a_1, a_2, \ldots) \cdot (b_0, b_1, b_2, \ldots) & = (c_0, c_1, c_2, \ldots)
	\end{align*}
	for all $(a_0, a_1, a_2, \ldots), (b_0, b_1, b_2, \ldots) \in \mathbb{Z}[x]$ where $c_n = \sum_{k = 0}^n a_k b_{n - k}$ for all $n \in \mathbb{Z}^{+}$. The unary operation $-$ on $\mathbb{Z}[x]$ is defined by $-(a_0, a_1, a_2, \ldots) = (-a_0, -a_1, -a_2, \ldots)$.

	The sequence $(a_0, a_1, a_2, \ldots)$ is also denoted $(a_n)_{n=0}^\infty$.
\end{definition}

\begin{definition}
	\xlabel[greatest element]{int2:e:7:d:greatest_element}
	Let $R$ be an \nameref{int2:d:oid}, and let $A \subseteq R$ be a set.
	The set $A$ has a \emph{\nameref*{int2:e:7:d:greatest_element}} if there is some $a \in A$ such that $a \geq x$ for all $x \in A$.
\end{definition}

\begin{definition}
	\xlabel[degree]{int2:e:7:d:degree}
	The \emph{\nameref*{int2:e:7:d:degree}} of $a \in \mathbb{Z}[x]$, denoted $\deg(a)$, is either $0$ if $a = \widetilde{0}$ or the greatest integer $n \in \mathbb{Z}^{+}$ such that $a_n \not= 0$.
\end{definition}

\begin{definition}
	\xlabel[dictionary order]{int2:e:7:d:dictionary_order}
	The relation $<$, called the \emph{\nameref*{int2:e:7:d:dictionary_order}} on $\mathbb{Z}[x]$, is defined by $a < b$ if and only if either $\deg(a) < \deg(b)$ or $\deg(a) = \deg(b)$ and there is the greatest ${n \leq \deg(a) = \deg(b)}$ such that $a_n < b_n$ for all $a, b \in \mathbb{Z}[x]$.
\end{definition}

\begin{proof}[(\ref{int2:e:7:1})]
	Let $a, b, c \in \mathbb{Z}[x]$. We will prove that $\mathbb{Z}[x]$ satisfies the following properties, which will imply that $\mathbb{Z}[x]$ is an \nameref{int2:d:oid}.
	\begin{enumerate}
		\item \xlabel[Associative Law for Addition]{int2:e:7:oid:associative_add}
		      $(a + b) + c = a + (b + c)$ \quad (\nameref*{int2:e:7:oid:associative_add}).
		\item \xlabel[Commutative Law for Addition]{int2:e:7:oid:commutative_add}
		      $a + b = b + a$ \quad (\nameref*{int2:e:7:oid:commutative_add}).
		\item \xlabel[Identity Law for Addition]{int2:e:7:oid:identity_add}
		      $a + \widetilde{0} = a$ \quad (\nameref*{int2:e:7:oid:identity_add}).
		\item \xlabel[Inverses Law for Addition]{int2:e:7:oid:inverses_add}
		      $a + (-a) = \widetilde{0}$ \quad (\nameref*{int2:e:7:oid:inverses_add}).
		\item \xlabel[Associative Law for Multiplication]{int2:e:7:oid:associative_mult}
		      $(a b) c = a (b c)$ \quad (\nameref*{int2:e:7:oid:associative_mult}).
		\item \xlabel[Commutative Law for Multiplication]{int2:e:7:oid:commutative_mult}
		      $a b = b a$ \quad (\nameref*{int2:e:7:oid:commutative_mult}).
		\item \xlabel[Identity Law for Multiplication]{int2:e:7:oid:identity_mult}
		      $a \cdot \widetilde{1} = a$ \quad (\nameref*{int2:e:7:oid:identity_mult}).
		\item \xlabel[Distributive Law]{int2:e:7:oid:distributive}
		      $a (b + c) = a b + a c$ \quad (\nameref*{int2:e:7:oid:distributive}).
		\item \xlabel[No Zero Divisors Law]{int2:e:7:oid:no_zero_divisors}
		      If $a b = \widetilde{0}$, then $a = \widetilde{0}$ or $b = \widetilde{0}$ \quad (\nameref*{int2:e:7:oid:no_zero_divisors}).
		\item \xlabel[Trichotomy Law]{int2:e:7:oid:trichotomy}
		      Precisely one of $a < b$ or $a = b$ or $a > b$ holds \quad (\nameref*{int2:e:7:oid:trichotomy}).
		\item \xlabel[Transitive Law]{int2:e:7:oid:transitive}
		      If $a < b$ and $b < c$, then $a < c$ \quad (\nameref*{int2:e:7:oid:transitive}).
		\item \xlabel[Addition Law for Order]{int2:e:7:oid:addition_order}
		      If $a < b$ then $a + c < b + c$ \quad (\nameref*{int2:e:7:oid:addition_order}).
		\item \xlabel[Multiplication Law for Order]{int2:e:7:oid:multiplication_order}
		      If $a < b$ and $c > \widetilde{0}$, then $a c < b c$ \quad (\nameref*{int2:e:7:oid:multiplication_order}).
		\item \xlabel[Non-Triviality]{int2:e:7:oid:non_triviality}
		      $\widetilde{0} \neq \widetilde{1}$ \quad (\nameref*{int2:e:7:oid:non_triviality}).
	\end{enumerate}

	\PartProof{int2:e:7:oid:associative_add}
	Using the \nameref{int2:d:oid:associative_add} for the integers we obtain
	\begin{align*}
		(a + b) + c & = [(a_0, a_1, a_2, \ldots) + (b_0, b_1, b_2, \ldots)] + (c_0, c_1, c_2, \ldots) \\
		            & = (a_0 + b_0, a_1 + b_1, a_2 + b_2, \ldots) + (c_0, c_1, c_2, \ldots)           \\
		            & = ([a_0 + b_0] + c_0,\; [a_1 + b_1] + c_1,\; [a_2 + b_2] + c_2,\; \ldots)       \\
		            & = (a_0 + [b_0 + c_0],\; a_1 + [b_1 + c_1],\; a_2 + [b_2 + c_2],\; \ldots)       \\
		            & = (a_0, a_1, a_2, \ldots) + (b_0 + c_0, b_1 + c_1, b_2 + c_2, \ldots)           \\
		            & = (a_0, a_1, a_2, \ldots) + [(b_0, b_1, b_2, \ldots) + (c_0, c_1, c_2, \ldots)] \\
		            & = a + (b + c).
	\end{align*}

	\PartProof{int2:e:7:oid:commutative_add}
	Using the \nameref{int2:d:oid:commutative_add} for the integers we obtain
	\begin{align*}
		a + b & = (a_0, a_1, a_2, \ldots) + (b_0, b_1, b_2, \ldots) \\
		      & = (a_0 + b_0,\; a_1 + b_1,\; a_2 + b_2,\; \ldots)   \\
		      & = (b_0 + a_0,\; b_1 + a_1,\; b_2 + a_2,\; \ldots)   \\
		      & = (b_0, b_1, b_2, \ldots) + (a_0, a_1, a_2, \ldots) \\
		      & = b + a.
	\end{align*}

	\PartProof{int2:e:7:oid:identity_add}
	Using the \nameref{int2:d:oid:identity_add} for the integers it follows that
	\begin{align*}
		a + \widetilde{0} & = (a_0, a_1, a_2, \ldots) + (0, 0, 0, \ldots) \\
		                  & = (a_0 + 0,\; a_1 + 0,\; a_2 + 0,\; \ldots)   \\
		                  & = (a_0, a_1, a_2, \ldots) = a.
	\end{align*}

	\PartProof{int2:e:7:oid:inverses_add}
	Using the \nameref{int2:d:oid:inverses_add} for the integers we get
	\begin{align*}
		a + (-a) & = (a_0, a_1, a_2, \ldots) + [-(a_0, a_1, a_2, \ldots)]     \\
		         & = (a_0, a_1, a_2, \ldots) + (-a_0, -a_1, -a_2, \ldots)     \\
		         & = (a_0 + [-a_0],\; a_1 + [-a_1],\; a_2 + [-a_2],\; \ldots) \\
		         & = (0, 0, 0, \ldots) = \widetilde{0}.
	\end{align*}

	\PartProof{int2:e:7:oid:associative_mult}
	Using the \nameref{int2:d:oid:distributive} for the integers we obtain
	\begin{align*}
		(a b) c & = \left( \sum_{i=0}^{n}(a b)_{i} c_{n-i} \right)_{n=0}^\infty = \left( \sum_{i=0}^{n} \sum_{j=0}^{i} a_{j} b_{i-j} c_{n-i}  \right)_{n=0}^\infty                    \\
		        & = \left( \sum_{j=0}^{n} \sum_{i=j}^{n} a_{j} b_{i-j} c_{n-i} \right)_{n=0}^\infty = \left( \sum_{j=0}^{n} a_{j} \sum_{i=j}^{n} b_{i-j} c_{n-i} \right)_{n=0}^\infty \\
		        & = \left( \sum_{j=0}^{n} a_{j} \sum_{k=0}^{n-j} b_{k} c_{n-j-k} \right)_{n=0}^\infty = \left( \sum_{j=0}^{n} a_{j}(b c)_{n-j} \right)_{n=0}^\infty                   \\
		        & = a(b c).
	\end{align*}

	\PartProof{int2:e:7:oid:commutative_mult}
	Using the \nameref{int2:d:oid:commutative_mult} for the integers we get
	$$
		a b = \left( \sum_{i=0}^n a_{i} b_{n-i} \right)_{n=0}^\infty = \left( \sum_{j=0}^n a_{n-j} b_{j} \right)_{n=0}^\infty = \left( \sum_{j=0}^n b_{j} a_{n-j} \right)_{n=0}^\infty = b a.
	$$


	\PartProof{int2:e:7:oid:identity_mult}
	By the definition of $\widetilde{1}$ we know that $\widetilde{1}_0 = 1$ and if $n \not= 0$ then $\widetilde{1}_n = 0$. Using Lemma \ref{int2:l:props} (\ref{int2:l:props:4}) and the \hyperref[int2:d:oid:identity_add]{Identity} and \hyperref[int2:d:oid:commutative_add]{Commutative} Laws for Addition for the integers it then follows that
	\begin{align*}
		a \cdot \widetilde{1} & = \left( \sum_{i=0}^n (a_i \cdot \widetilde{1}_{n-i}) \right)_{n=0}^\infty
		= \left( \sum_{i=0}^{n-1} (a_i \cdot \widetilde{1}_{n-i}) + a_n \cdot \widetilde{1}_0 \right)_{n=0}^\infty \\
		\
		                      & = \left( \sum_{i=0}^{n-1} (a_i \cdot 0) + a_n \cdot 1 \right)_{n=0}^\infty
		= \left( \sum_{i=0}^{n-1} 0 + a_n \right)_{n=0}^\infty                                                     \\
		\
		                      & = (0 + a_n)_{n=0}^\infty = (a_n + 0)_{n=0}^\infty = (a_n)_{n=0}^\infty = a.
	\end{align*}

	\PartProof{int2:e:7:oid:distributive}
	Using the \nameref{int2:d:oid:distributive} for the integers we obtain
	\begin{align*}
		a(b + c) & = \left( \sum_{i=0}^n a_i(b + c)_{n-i} \right)_{n=0}^\infty = \left( \sum_{i=0}^n a_i(b_{n-i} + c_{n-i}) \right)_{n=0}^\infty                            \\
		         & = \left( \sum_{i=0}^n (a_i b_{n-i} + a_i c_{n-i}) \right)_{n=0}^\infty = \left( \sum_{i=0}^n a_i b_{n-i} + \sum_{i=0}^n a_i c_{n-i} \right)_{n=0}^\infty \\
		         & = \left( \sum_{i=0}^n a_i b_{n-i} \right)_{n=0}^\infty + \left( \sum_{i=0}^n a_i c_{n-i} \right)_{n=0}^\infty = a b + a c.
	\end{align*}

	\PartProof{int2:e:7:oid:no_zero_divisors}
	Suppose that $a b = \widetilde{0}$, and suppose to the contrary that $a \not= \widetilde{0}$ and $b \not= \widetilde{0}$. By the definition of polynomial we know that there is some $p, q \in \mathbb{Z}^{+}$ such that $a_p$ and $b_q$ are \hyperref[int2:e:7:d:greatest_element]{greatest} nonzero elements of $a$ and $b$, respectively. This implies $a_x = 0 = b_y$ for all $x > p$ and $y > q$. Then,
	\begin{align*}
		\sum_{i=0}^{p+q} a_i b_{p+q-i} & = \sum_{i=0}^{p-1} a_i b_{p+q-i} + a_p b_q + \sum_{i=p+1}^{p+q-i} a_i b_{p+q-i}    \\
		                               & = \sum_{i=0}^{p-1} a_i \cdot 0 + a_p b_q + \sum_{i=p+1}^{p+q-i} 0 \cdot b_{p+q-i}.
	\end{align*}
	By the \nameref{int2:d:oid:commutative_mult} for the integers we observe that
	$$
		\sum_{i=p+1}^{p+q-i} 0 \cdot b_{p+q-i} = \sum_{i=p+1}^{p+q-i} b_{p+q-i} \cdot 0,
	$$
	and from the \nameref{int2:d:oid:identity_mult} and the \nameref{int2:d:oid:identity_add} for the integers we deduce that
	$$
		\sum_{i=0}^{p-1} a_i \cdot 0 = \sum_{i=0}^{p-1} 0 = 0 = \sum_{i=p+1}^{p+q-i} 0 = \sum_{i=p+1}^{p+q-i} b_{p+q-i} \cdot 0.
	$$
	Finally, by repeated use of the \hyperref[int2:d:oid:commutative_add]{Commutative} and \hyperref[int2:d:oid:identity_add]{Identity} Laws for Addition for the integers we obtain
	$$
		\sum_{i=0}^{p+q} a_i b_{p+q-i} = 0 + a_p b_q + 0 = (a_p b_q + 0) + 0 = a_p b_q + 0 = a_p b_q.
	$$
	Since $a_p \not= 0$ and $b_q \not= 0$, the \nameref{int2:d:oid:no_zero_divisors} for the integers implies that $a_p b_q \not= 0$. Hence, $\sum_{i=0}^{p+q} a_i b_{p+q-i} \not= 0$. But then $\left( \sum_{i=0}^n a_i b_{n-i} \right)_{n=0}^\infty \not= \widetilde{0}$, which is a contradiction to the fact that $a b = \widetilde{0}$. Hence, there is either $a = \widetilde{0}$ or $b = \widetilde{0}$.

	\PartProof{int2:e:7:oid:trichotomy}
	Suppose that $a = b$. Then $(a_n)_{n=0}^\infty = (b_n)_{n=0}^\infty$. Because of the \nameref{int2:d:oid:trichotomy} for the integers we deduce that $a_n \not< b_n$ and $a_n \not> b_n$ for all $n \in \mathbb{Z}^{+}$. Hence, $a \not< b$ and $a \not> b$.

	Now let $n = \deg(a)$ and $m = \deg(b)$. Without loss of generality, suppose that $a < b$. Suppose that $n < m$. By the \nameref{int2:d:oid:trichotomy} for the integers it follows that $n \not= m$ and $n \not> m$, and hence by the definition of dictionary order we can conclude that $a \not= b$ and $a \not> b$. We consider the following two cases.
	\begin{bycases}
		\item $n < m$. By the \nameref{int2:d:oid:trichotomy} for the integers it follows that $n \not= m$ and $n \not> m$, and hence by the definition of dictionary order we can conclude that $a \not= b$ and $a \not> b$.
		\item $n \not> m$. Then $n = m$ and there is the greatest $p \leq n = m$ such that $a_p < b_p$. By the \nameref{int2:d:oid:trichotomy} for the integers we see that $a_p \not= b_p$ and $a_p \not> b_p$. Hence by the definition of dictionary order we have shown that $a \not= b$ and $a \not> b$.
	\end{bycases}

	Finally, suppose to the contrary that $a \not= b$ and $a \not< b$ and $a \not> b$. Then $n \not< m$ and $n \not= m$ and $n \not> m$, which is a contradiction to the \nameref{int2:d:oid:trichotomy} for the integers. Hence, precisely one of $a < b$ or $a = b$ or $a > b$ holds.

	\PartProof{int2:e:7:oid:transitive}
	Let $p$, $q$, and $r$ be the \hyperref[int2:e:7:d:degree]{degrees} of $a$, $b$, and $c$, respectively. If $p < q$ and $q < r$, then by the \nameref{int2:d:oid:trichotomy} for the integers we get $p < r$, and hence $a < c$. Suppose that $p = q = r$. There is the greatest $s \leq p = q = r$ and $t \leq p = q = r$ such that $a_{s} < b_{s}$ and $b_{t} < c_{t}$. We consider the following three cases.
	\begin{bycases}
		\item $s < t$. Then $a_s < b_s = c_s$, and hence $a < c$.
		\item $s > t$. Then $a_t = b_t < c_t$, and hence $a < c$.
		\item $s = t$. Then $a_s < b_s < c_s$, and by the \nameref{int2:d:oid:trichotomy} for the integers we deduce that $a_s < c_s$. Hence, $a < c$.
	\end{bycases}

	Now suppose, without loss of generality, that $p < q$ and $q = r$. Then clearly $p < r$, and as a result $a < c$.

	\PartProof{int2:e:7:oid:addition_order}
	Let $d_a, d_b, d_c, d_{a+c}, d_{b+c} \in \mathbb{Z}^{+}$ be the degrees of $a, b, c, a+c, b+c \in \mathbb{Z}[x]$, respectively. We note that $d_{a+c} = \max(d_a, d_c)$ and $d_{b+c} = \max(d_b, d_c)$. Suppose that $d_a = d_b$. Then $\max(d_a, d_c) = \max(d_b, d_c)$, and then $d_{a+c} = d_{b+c}$. Also, we observe that there is the greatest $p \leq d_a = d_b$ such that $a_p < b_p$. By \nameref{int2:d:oid:addition_order} for the integers it follows that $a_p + c_p < b_p + c_p$. Hence, $a + c < b + c$.

	Now suppose that $d_a < d_b$. We consider the following three cases.
	\begin{bycases}
		\item $d_a < d_b < d_c$. Then $\max(d_a, d_c) = d_c$ and $\max(d_b, d_c) = d_c$, and then $d_{a+c} = d_{b+c}$. Let $p = d_b$. Because $d_a < d_b$ we see that $p \leq d_{a+c} = d_{b+c}$ is the greatest element of $\mathbb{Z}^{+}$ such that $a_p < b_p$.  By \nameref{int2:d:oid:addition_order} for the integers it follows that $a_p + c_p < b_p + c_p$, and hence $a + c < b + c$.
		\item $d_c \leq d_a < d_b$. Then $\max(d_a, d_c) = d_a$ and $\max(d_b, d_c) = d_b$, and then $d_{a+c} < d_{b+c}$. Hence $a + c < b + c$.
		\item $d_a < d_c \leq d_b$. A similar argument shows that $a + c < b + c$.
	\end{bycases}

	\PartProof{int2:e:7:oid:multiplication_order}
	We note that $\deg(a b) = \deg(a) + \deg(b)$ for all $a, b \in \mathbb{Z}[x]$. Suppose that $a < b$ and $c > \widetilde{0}$. Suppose further that $\deg(a) < \deg(b)$. Because $c > \widetilde{0}$ it follows that $\deg(c) > 0$. We then deduce from the \nameref{int2:d:oid:addition_order} for the integers that $\deg(a) + \deg(c) < \deg(b) + \deg(c)$. Then $\deg(a c) < \deg(b c)$, and hence $a c < b c$.

	Now suppose that $\deg(a) = \deg(b)$. Then there is the greatest $p \leq \deg(a) = \deg(b)$ such that $a_p < b_p$. Then $\deg(a c) = \deg(b c)$, and because $c > \widetilde{0}$ we have $p \leq \deg(a c) = \deg(b c)$. Also, by the \nameref{int2:d:oid:multiplication_order} for the integers we see that $a_p c_p < b_p c_p$, and hence $a c < b c$.

	\PartProof{int2:e:7:oid:non_triviality}
	Because $\widetilde{0}_0 \not= \widetilde{1}_0$ we can conclude that $\widetilde{0} \not= \widetilde{1}$.
\end{proof}

\begin{proof}[(\ref{int2:e:7:2})]
	Suppose to the contrary that $g \in \mathbb{Z}[x]$ such that $f < g < f + \widetilde{1}$. Suppose that $\deg(f) < \deg(g) < \deg(f + \widetilde{1})$. We can notice that $\deg(f + \widetilde{1}) = \deg(f)$, so $\deg(f) < \deg(g)$ and $\deg(g) < \deg(f)$, which is a contradiction to \nameref{int2:d:oid:trichotomy} for the integers. Hence, either $f \not< g$ or $g \not< f + \widetilde{1}$.

	Now suppose that $\deg(f) = \deg(g) = \deg(f + \widetilde{1})$. Then there is the greatest
	$$
		p \leq \deg(f) = \deg(g) = \deg(f + \widetilde{1})
	$$
	such that
	$$
		f_p < g_p < (f + \widetilde{1})_p = f_p + \widetilde{1}_p.
	$$
	If $p = 0$ then $f_0 < g_0 < f_0 + 1$, which is a contradiction to Theorem \ref{int2:t:discrete}. But if $p > 0$ then $f_p < g_p$ and $g_p < f_p$, which is a contradiction to the \nameref{int2:d:oid:trichotomy} for the integers. Hence, either $f \not< g$ or $g \not< f + \widetilde{1}$.

	Finally, suppose that either
	$$
		\deg(f) < \deg(g) = \deg(f + \widetilde{1}) \text{ or } \deg(f) = \deg(g) < \deg(f + \widetilde{1}).
	$$
	But we know that ${\deg(f) = \deg(f + \widetilde{{1}}})$, so this implies that either $\deg(f) < \deg(g)$ and $\deg(f) = \deg(g)$ or $\deg(f) = \deg(g)$ and $\deg(g) < \deg(f)$, which is a contradiction to the \nameref{int2:d:oid:trichotomy} for the integers. Hence, either $f \not< g$ or $g \not< f + \widetilde{1}$.
\end{proof}

\begin{proof}[(\ref{int2:e:7:3})]
	Let $A = \{ a \in \mathbb{Z}[x] \mid \deg(a) = 1 \}$. Clearly, $A$ is nonempty, and since $\deg(0) = 0$ we see that $a > \widetilde{0}$ for all $a \in A$. Suppose to the contrary that there is some $a' \in A$ such that $a' \leq a$ for all $a \in A$. Because $\deg(a) = 1$ for all $a \in A$, there is the greatest $p \in \mathbb{Z}^{+}$ such that $a'_p \leq a_p$ for all $a \in A$. Let $b = a' + [(-a'_p) + (-a'_p)]x$, and as a result $b_p = a'_p + [(-a'_p) + (-a'_p)]$. By repeated use of the \hyperref[int2:d:oid:inverses_add]{Inverses} and \hyperref[int2:d:oid:identity_add]{Identity} Laws for Addition for the integers we obtain $b_p = -a'_p$. Then $\deg(b_p) = 1$, and hence $b \in A$. By \nameref{int2:d:oid:transitive} for the integers, $-a'_p < 0 < a_p$ implies $-a'_p = b_p < a'_p$. From the \nameref{int2:d:oid:trichotomy} for the integers we then deduce that $a' \not\leq b$, which is a contradiction to the fact that $a' \leq a$ for all $a \in A$. Hence, $\mathbb{Z}[x]$ does not satisfy the \nameref{int2:d:wop}.
\end{proof}


%-------------------------------------------------------------------------------------------------------------
\Newpage
\intiiLessRelL*

\begin{exercise} %% 1.4.8
	\label{int2:e:8}
	Let $a \in \mathbb{Z}$.
	\begin{enumerate}
		\item \label{int2:e:8:1}
		      Let $G \subseteq \{ x \in \mathbb{Z} \mid x \geq a \}$ be a set. Suppose that $a \in G$, and that if $g \in G$ then $g + 1 \in G$. Prove that $G = \{ x \in \mathbb{Z} \mid x \geq a \}$.
		      \hfill [Use Exercise \ref{int2:e:4}.]
		\item \label{int2:e:8:2}
		      Let $H \subseteq \{x \in \mathbb{Z} \mid x \leq a \}$ be a set. Suppose that $a \in H$, and that if $h \in H$ then $h + (-1) \in H$. Prove that $H = \{ x \in \mathbb{Z} \mid x \leq a \}$.
		      \hfill [Use Exercise \ref{int2:e:3}.]
	\end{enumerate}

\end{exercise}

\begin{proof}[(\ref{int2:e:8:1})]
	Let $y \in \mathbb{Z}$, and suppose that $y \geq a$. If $y = a$, then clearly $y \in G$. Now suppose that $y \not= a$, so $y > a$. Then according to Lemma\,\ref{int2:l:less_relation} we can choose some $k \in \mathbb{N}$ such that $y = a + k$.

	Let $I = \{ n \in \mathbb{N} \mid a + n \in G \}$. Clearly $I \subseteq \mathbb{N}$. Because $a \in G$ it follows that $a + 1 \in G$, and therefore $1 \in I$. Now suppose that $n \in I$, which means that $a + n \in G$. Then $(a + n) + 1 \in G$. By the \nameref{int2:d:oid:associative_add} we have $a + (n + 1) \in G$. Hence $n + 1 \in I$, and using Part (\ref{int2:t:peano:induction}) of the \nameref{int2:t:peano} we deduce that $I = \mathbb{N}$. Hence, $y = a + k \in G$.
\end{proof}

\begin{proof}[(\ref{int2:e:8:2})]
	Let $y \in \mathbb{Z}$, and suppose that $y \leq a$. By Exercise \ref{int2:e:3} it follows that ${-y \geq -a}$, which means that either $-y = -a$ or $-y > -a$. Suppose that $-y = -a$. Then by the \nameref{int2:d:oid:identity_mult} we have $-(y \cdot 1) = -(a \cdot 1)$. By Lemma \ref{int2:l:props} (\ref{int2:l:props:6}) and the \nameref{int2:l:props:cancellation_mult} we deduce that $y = a$. Hence, $y \in H$.

	Now suppose that $-y > -a$. It follows from Lemma \ref{int2:l:less_relation} that there is some ${k \in \mathbb{N}}$ such that $-y = (-a) + k$. Multiplying both sides of this equation by $-1$ we obtain ${(-1)(-y) = (-1)[(-a) + k]}$. According to Lemma \ref{int2:l:props} (\ref{int2:l:props:6}) and the \nameref{int2:d:oid:commutative_mult} we obtain $[-(-y)] \cdot 1 = [-(-a)] \cdot 1 + (-k) \cdot 1$, and by Lemma \ref{int2:l:props} (\ref{int2:l:props:2}) and the \nameref{int2:d:oid:identity_mult} we see that $y = a + (-k)$.

	Let $I = \{ n \in \mathbb{N} \mid a + (-n) \in H \}$. Clearly $I \subseteq H$. Because $a \in G$ it follows that $a + (-1) \in G$, and therefore $1 \in I$. Now suppose that $n \in I$. Then $a + (-n) \in H$, so $[a + (-n)] + (-1) \in H$. Using the \nameref{int2:d:oid:associative_add} we get $a + [(-n) + (-1)]$, and using Part (\ref{int2:l:props:3}) of Lemma \ref{int2:l:props} we get $a + [-(n + 1)] \in H$. Thus, $n + 1 \in I$, and by Part (\ref{int2:t:peano:induction}) of the \nameref{int2:t:peano} we deduce that $I = \mathbb{N}$. Hence, $y = a + (-k) \in H$.
\end{proof}
