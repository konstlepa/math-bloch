%=============================================================================================================
%= SECTION 1.7
%=============================================================================================================
\section{Constructing the Real Numbers}
\label{real}

\begin{definition} %% 1.7.1
	The set of \emph{real numbers}, denoted $\mathbb{R}$, is defined by
	$$
		\mathbb{R} = \{ A \subseteq \mathbb{Q} \mid A \text { is a Dedekind cut } \}.
	$$
\end{definition}

\begin{definition} %% 1.7.2
	The relation $<$ on $\mathbb{R}$ is defined by $A < B$ if and only if $A \supsetneq B$, for all $A, B \in \mathbb{R}$. The relation $\leq$ on $\mathbb{R}$ is defined by $A \leq B$ if and only if $A \supseteq B$, for all $A, B \in \mathbb{R}$.
\end{definition}

\begin{definition} %% 1.7.3
	\label{real:d:add}
	The binary operation $+$ on $\mathbb{R}$ is defined by
	$$
		A + B = \{ r \in \mathbb{Q} \mid r = a + b \text { for some } a \in A \text { and } b \in B \}
	$$
	for all $A, B \in \mathbb{R}$. The unary operation $-$ on $\mathbb{R}$ is defined by
	$$
		-A = \{ r \in \mathbb{Q} \mid -r < c \text { for some } c \in \mathbb{Q}-A \}
	$$
	for all $A \in \mathbb{R}$.
\end{definition}

\begin{lemma} %% 1.7.4
	\label{real:l:mult_req}
	Let $A \in \mathbb{R}$, and let $r \in \mathbb{Q}$.
	\begin{enumerate}
		\item \label{real:l:mult_req:1}
		      $A > D_r$ if and only if there is some $q \in \mathbb{Q}-A$ such that $q > r$.
		\item \label{real:l:mult_req:2}
		      $A \geq D_r$ if and only if $r \in \mathbb{Q}-A$ if and only if $a > r$ for all $a \in A$.
		\item \label{real:l:mult_req:3}
		      If $A < D_0$ then $-A \geq D_0$.
	\end{enumerate}
\end{lemma}

\begin{definition} %% 1.7.5
	\label{real:d:mult}
	The binary operation $\cdot$ on $\mathbb{R}$ is defined by
	$$
		A \cdot B =
		\begin{cases}
			\{ r \in \mathbb{Q} \mid r = a b \text { for some } a \in A \text { and } b \in B \}, & \text{ if } A \geq D_0 \text { and } B \geq D_0 \\
			-[(-A) \cdot B],                                                                      & \text{ if } A < D_0 \text { and } B \geq D_0    \\
			-[A \cdot(-B)],                                                                       & \text{ if } A \geq D_0 \text { and } B < D_0    \\
			(-A) \cdot(-B),                                                                       & \text{ if } A < D_0 \text { and } B < D_0.
		\end{cases}
	$$
	The unary operation ${ }^{-1}$ on $\mathbb{R}-\left\{ D_0 \right\}$ is defined by
	$$
		A^{-1} =
		\begin{cases}
			\{ r \in \mathbb{Q} \mid r > 0 \text { and } \frac{1}{r} < c \text { for some } c \in \mathbb{Q}-A \}, & \text { if } A > D_0  \\
			-(-A)^{-1},                                                                                            & \text { if } A < D_0.
		\end{cases}
	$$
\end{definition}

\begin{theorem} %% 1.7.6
	\label{real:t:props}
	Let $A, B, C \in \mathbb{R}$.
	\begin{enumerate}
		\item $(A + B) + C = A + (B + C)$ \quad (\nameref*{real:t:props:associative_add}). \xlabel[Associative Law for Addition]{real:t:props:associative_add}
		\item $A + B = B + A$ \quad (\nameref*{real:t:props:commutative_add}). \xlabel[Commutative Law for Addition]{real:t:props:commutative_add}
		\item $A + D_0 = A$ \quad (\nameref*{real:t:props:identity_add}). \xlabel[Identity Law for Addition]{real:t:props:identity_add}
		\item $A + (-A) = D_0$ \quad (\nameref*{real:t:props:inverses_add}). \xlabel[Inverses Law for Addition]{real:t:props:inverses_add}
		\item $(AB)C = A(BC)$ \quad (\nameref*{real:t:props:associative_mult}). \xlabel[Associative Law for Multiplication]{real:t:props:associative_mult}
		\item $A B = B A$ \quad (\nameref*{real:t:props:commutative_mult}). \xlabel[Commutative Law for Multiplication]{real:t:props:commutative_mult}
		\item $A \cdot D_1 = A$ \quad (\nameref*{real:t:props:identity_mult}). \xlabel[Identity Law for Multiplication]{real:t:props:identity_mult}
		\item If $A \neq D_0$, then $A A^{-1} = D_1$ \quad (\nameref*{real:t:props:inverses_mult}). \xlabel[Inverses Law for Multiplication]{real:t:props:inverses_mult}
		\item $A(B + C) = A B + A C$ \quad (\nameref*{real:t:props:distributive}). \xlabel[Distributive Law]{real:t:props:distributive}
		\item Precisely one of $A < B$ or $A = B$ or $A > B$ holds \quad (\nameref*{real:t:props:trichotomy}). \xlabel[Trichotomy Law]{real:t:props:trichotomy}
		\item If $A < B$ and $B < C$, then $A < C$ \quad (\nameref*{real:t:props:transitive}). \xlabel[Transitive Law]{real:t:props:transitive}
		\item If $A < B$ then $A + C < B + C$ \quad (\nameref*{real:t:props:addition_order}). \xlabel[Addition Law for Order]{real:t:props:addition_order}
		\item If $A < B$ and $C > D_0$, then $A C < B C$ \quad (\nameref*{real:t:props:multiplication_order}). \xlabel[Multiplication Law for Order]{real:t:props:multiplication_order}
		\item $D_0 < D_1$ \quad (\nameref*{real:t:props:non_triviality}). \xlabel[Non-Triviality]{real:t:props:non_triviality}
	\end{enumerate}
\end{theorem}

\begin{definition} %% 1.7.7
	Let $A \subseteq \mathbb{R}$ be a set.
	\begin{enumerate}
		\item The set $A$ is \emph{bounded above} if there is some $M \in \mathbb{R}$ such that $X \leq M$ for all $X \in A$. The number $M$ is called an \emph{upper bound} of $A$.
		\item The set $A$ is \emph{bounded below} if there is some $P \in \mathbb{R}$ such that $X \geq P$ for all $X \in A$. The number $P$ is called a \emph{lower bound} of $A$.
		\item The set $A$ is \emph{bounded} if it is bounded above and bounded below.
		\item Let $M \in \mathbb{R}$. The number $M$ is a \emph{least upper bound} (also called a \emph{supremum}) of $A$ if $M$ is an upper bound of $A$, and if $M \leq T$ for all upper bounds $T$ of $A$.
		\item Let $P \in \mathbb{R}$. The number $P$ is a \emph{greatest lower bound} (also called an \emph{infimum}) of $A$ if $P$ is a lower bound of $A$, and if $P \geq V$ for all lower bounds $V$ of $A$.
	\end{enumerate}
\end{definition}

\begin{theorem}[Greatest Lower Bound Property] %% 1.7.8
	Let $A \subseteq \mathbb{R}$ be a set. If $A$ is non-empty and bounded below, then $A$ has a greatest lower bound.
\end{theorem}

\begin{theorem}[Least Upper Bound Property] %% 1.7.9
	Let $A \subseteq \mathbb{R}$ be a set. If $A$ is nonempty and bounded above, then A has a least upper bound.
\end{theorem}

\begin{theorem} %% 1.7.10
	\label{real:t:rat}
	Let $i: \mathbb{Q} \to \mathbb{R}$ be defined by $i(r) = D_r$ for all $r \in \mathbb{R}$.
	\begin{enumerate}
		\item \label{real:t:rat:1}
		      The function $i: \mathbb{Q} \to \mathbb{R}$ is injective.
		\item \label{real:t:rat:2}
		      $i(0) = D_0$ and $i(1) = D_1$.
		\item \label{real:t:rat:3}
		      Let $r, s \in \mathbb{Q}$. Then
		      \begin{enumerate}
			      \item \label{real:t:rat:3:1}
			            $i(r + s)=i(r) + i(s)$;
			      \item \label{real:t:rat:3:2}
			            $i(-r) = -i(r)$;
			      \item \label{real:t:rat:3:3}
			            $i(r s) = i(r) i(s)$;
			      \item \label{real:t:rat:3:4}
			            if $r \neq 0$ then $i(r^{-1}) = [i(r)]^{-1}$;
			      \item \label{real:t:rat:3:5}
			            $r < s$ if and only if $i(r) < i(s)$.
		      \end{enumerate}
	\end{enumerate}
\end{theorem}


%-------------------------------------------------------------------------------------------------------------
\Newpage
\begin{exercise}[Used in Exercise \ref{real:e:7}] %% 1.7.1
	\label{real:e:1}
	Let $r \in \mathbb{Q}$.
	\begin{enumerate}
		\item \label{real:e:1:1}
		      Prove that $D_{-r} = -D_r$, using only Definition \ref{cuts:d:rat_irrat} and Definition \ref{real:d:add}.
		\item \label{real:e:1:2}
		      Prove that $D_{r^{-1}} = [D_r]^{-1}$, using only Definition \ref{real:d:mult} and Definition \ref{real:d:add}.
	\end{enumerate}
\end{exercise}

\begin{proof}[(\ref{real:e:1:1})]
	Let $x \in \mathbb{Q}$. By the definition of $D_{-r}$ we see that $-x < r$. Because $r \notin D_r$ we get $r \in \mathbb{Q} - D_{r}$, and by the definition of $-D_r$ we deduce that $x \in -D_r$. Hence, $D_{-r} \subseteq -D_r$. This process can be done backwards, and hence $D_{-r} = -D_r$.
\end{proof}

\begin{proof}[(\ref{real:e:1:2})]
	We note that $r \neq 0$ because of the definition of ${ }^{-1}$ on $\mathbb{R} - \{ D_0 \}$. Suppose that $D_r > D_0$. Let $x \in \mathbb{Q}$. From the definition of $D_{r^{-1}}$ we observe that $x > r^{-1} = \frac{1}{r}$. Since $D_r > D_0$, from Lemma \ref{real:l:mult_req} (\ref{real:l:mult_req:2}) we see that $0 \in \mathbb{Q} - D_0$ implies $r > 0$. Then $x > 0$ and $\frac{1}{x} < r$. Hence, $x \in [D_r]^{-1}$. Thus, $D_{r^{-1}} \subseteq [D_r]^{-1}$, and because this process can be done backwards we deduce that $D_{r^{-1}} = [D_r]^{-1}$. Since $r$ was arbitrary, we can conclude that $D_{q^{-1}} = [D_q]^{-1}$ if $D_q > D_0$ for all $q \in \mathbb{Q}$.

	Now suppose that $D_r < D_0$. By Lemma \ref{real:l:mult_req} (\ref{real:l:mult_req:3}) it follows that $-D_r > D_0$. From Part\,(\ref{real:e:1:1}) of this exercise we know that $-D_q = D_{-q}$ for all $q \in \mathbb{Q}$, so $D_{-r} > D_0$. Then
	$$
		D_{r^{-1}} = D_{-[-(r^{-1})]} = -D_{-(-r^{-1})} = -D_{(-r)^{-1}} = -[D_{-r}]^{-1} = -(-D_r)^{-1} = [D_r]^{-1},
	$$
	as required.
\end{proof}


%-------------------------------------------------------------------------------------------------------------
\Newpage
\begin{exercise}[Used in Theorem \ref{real:t:props}] %% 1.7.2
	\label{real:e:2}
	Let $A, B \in \mathbb{R}$. Suppose that $A > D_0$ and $B > D_0$. For this exercise, you may use only results prior to Theorem \ref{real:t:props}.
	\begin{enumerate}
		\item \label{real:e:2:1}
		      Prove that $A B > D_0$.
		\item \label{real:e:2:2}
		      Prove that $A^{-1} > D_0$.
	\end{enumerate}
\end{exercise}

\begin{proof}[(\ref{real:e:2:1})]
	Because $D_0 - A \neq \emptyset$ and $D_0 - B \neq \emptyset$, it follows from the definition of $A B$ that $A B \neq D_0$. Now let $r \in A B$. Then $r = a b$ for some $a \in A$ and $b \in B$. Since $A > D_0$ and $B > D_0$, we deduce that $a > 0$ and $b > 0$. But then $r = a b > 0$. Hence $A B \subsetneq D_0$, or in other words $A B > D_0$.
\end{proof}

\begin{proof}[(\ref{real:e:2:2})]
	We note that $A^{-1} \neq D_0$ by the definition of $A^{-1}$. Let $r \in A^{-1}$. Then $r > 0$, and as a result $r \in D_0$. Hence $A^{-1} \subsetneq D_0$, or in other words $A^{-1} > D_0$.
\end{proof}


%-------------------------------------------------------------------------------------------------------------
\Newpage
\begin{exercise}[Used in Theorem \ref{real:t:props}] %% 1.7.3
	\label{real:e:3}
	Prove Theorem \ref{real:t:props} (\ref{real:t:props:non_triviality}). For this exercise, you may use only results prior to Theorem \ref{real:t:props}.

	\hfill [Use Exercise \ref{rat:e:6} (\ref{rat:e:6:1}).]
\end{exercise}

\begin{proof}
	We note that $D_0 \neq D_1$ because $1 \in D_0$ and $1 \notin D_1$. Let $x \in D_0$. Then $x > 1$. From Exercise \ref{rat:e:6} (\ref{rat:e:6:1}) we see that $x > 1 > 0$. Then $x > 0$, and as a result $x \in D_0$. Hence $D_1 \subsetneq D_0$, or in other words $D_0 < D_1$.
\end{proof}


%-------------------------------------------------------------------------------------------------------------
\Newpage
\begin{exercise}[Used in Theorem \ref{real:t:props}, Exercise \ref{real:e:5} and Exercise \ref{real:e:6}] %% 1.7.4
	\label{real:e:4}
	For this exercise, use only the properties of real numbers stated in Theorem \ref{real:t:props} (\ref{real:t:props:associative_add}) (\ref{real:t:props:commutative_add}) (\ref{real:t:props:identity_add}) (\ref{real:t:props:inverses_add}) (\ref{real:t:props:trichotomy}) (\ref{real:t:props:transitive}) (\ref{real:t:props:addition_order}) (\ref{real:t:props:non_triviality}); it is not necessary to use the definition of real numbers as \nameref{cuts:d:cuts}. Let $A, B \in \mathbb{R}$.
	\begin{enumerate}
		\item \label{real:e:4:1}
		      Prove that $A > D_0$ if and only if $-A < D_0$, and that $A < D_0$ if and only if $-A > D_0$.
		\item \label{real:e:4:2}
		      Prove that $-(-A) = A$.
		\item \label{real:e:4:3}
		      Prove that $-(A + B) = (-A) + (-B)$.
		\item \label{real:e:4:4}
		      Prove that if $A > D_0$ and $B > D_0$, then $A + B > D_0$, and that if $A < D_0$ and $B < D_0$, then $A + B < D_0$.
		\item \label{real:e:4:5}
		      Prove that $A = (-B) + (A + B) = B + [A + (-B)]$ and $-A = B + [-(B + A)]$.
	\end{enumerate}
\end{exercise}

\begin{proof}[(\ref{real:e:4:1})]
	Suppose that $A > D_0$. Let $x \in D_0$. By the definition of $D_0$, we see that $-x < 0$. From Lemma \ref{real:l:mult_req} (\ref{real:l:mult_req:2}) it follows that $0 \in \mathbb{Q} - A$, and hence $x \in -A$, which means that $D_0 \subseteq -A$. Because $D_0 - A \not= \emptyset$ we can find some $c \in D_0$ such that $c \notin A$. Then $c \in \mathbb{Q} - A$ and $0 = -0 < c$, and as a result $0 \in -A$. But we know that $0 \notin D_0$, so $-A \neq D_0$, and hence $-A < D_0$.

	Now we will prove that if $-A < D_0$, then $A > D_0$. Taking the contrapositive, suppose that $A \leq D_0$. If $A < D_0$ then from Lemma\,\ref{real:l:mult_req}\,(\ref{real:l:mult_req:3}) we deduce that $-A \geq D_0$. Suppose further that $A = D_0$. Then,
	\begin{align*}
		-A & = \{ r \in \mathbb{Q} \mid -r < c \text{ for some } c \in \mathbb{Q} - A \}                 \\
		   & = \{ r \in \mathbb{Q} \mid -r < 0 \text{ where } 0 \in \mathbb{Q} - D_0 = \mathbb{Q} - A \} \\
		   & = -D_0.
	\end{align*}
	Using Exercise \ref{real:e:1} (\ref{real:e:1:1}) we obtain $-A = -D_0 = D_{-0} = D_0$, and hence $-A \geq D_0$.

	A similar argument shows that $A < D_0$ if and only if $-A > D_0$.
\end{proof}

\begin{proof}[(\ref{real:e:4:2})]
	By the \nameref{real:t:props:inverses_add} we know that $A + (-A) = D_0$. Then,
	\begin{align*}
		D_0 & = A + (-A)                                                                                  \\
		    & \iff D_0 + (-(-A)) = A + (-A) + (-(-A))                                                     \\
		    & \iff D_0 + (-(-A)) = A + [(-A) + (-(-A))] & \text{(\nameref{real:t:props:associative_add})} \\
		    & \iff D_0 + (-(-A)) = A + D_0              & \text{(\nameref{real:t:props:inverses_add})}    \\
		    & \iff (-(-A)) + D_0 = A + D_0              & \text{(\nameref{real:t:props:commutative_add})} \\
		    & \iff -(-A) = A,
	\end{align*}
	as required.
\end{proof}

\begin{proof}[(\ref{real:e:4:3})]
	By the \nameref{real:t:props:identity_add} we observe that $D_0 = D_0 + D_0$. From the \nameref{real:t:props:inverses_add} we deduce that $(A + B) + [-(A + B)] = D_0$, that $A + (-A) = D_0$, and that $B + (-B) = D_0$. Then,
	$$
		(A + B) + [-(A + B)] = [A + (-A)] + [B + (-B)].
	$$
	By repeated use of the \hyperref[real:t:props:associative_add]{Associative} and \hyperref[real:t:props:commutative_add]{Commutative} Laws for Addition we obtain
	$$
		[-(A + B)] + (A + B) = (-A) + (-B) + (A + B).
	$$
	Adding $-(A + B)$ to both sides of this equation and using the \hyperref[real:t:props:associative_add]{Associtative}, \hyperref[real:t:props:inverses_add]{Inverses} and \hyperref[real:t:props:identity_add]{Identity} Laws for Addition it follows that
	$$
		-(A + B) = (-A) + (-B),
	$$
	as required.
\end{proof}

\begin{proof}[(\ref{real:e:4:4})]
	Suppose that $A > D_0$ and $B > D_0$. From the \nameref{real:t:props:addition_order} we deduce that $A + B > D_0 + B$. By the \hyperref[real:t:props:commutative_add]{Commutative} and \hyperref[real:t:props:identity_add]{Identity} Laws for Addition we see that $D_0 + B = B + D_0 = B$. Then $A + B > B > D_0$, and hence by the \nameref{rat:t:props:transitive} $A + B > D_0$. A similar argument shows that if $A < D_0$ and $B < D_0$, then $A + B < D_0$.
\end{proof}

\begin{proof}[(\ref{real:e:4:5})]
	By the \nameref{real:t:props:identity_add} we observe that $A = A + D_0$. Then,
	\begin{align*}
		A & = A + D_0 = A + [B + (-B)] & \text{(\nameref{real:t:props:inverses_add})}     \\
		  & = (A + B) + (-B)           & \text{(\nameref{real:t:props:associative_add})}  \\
		  & = (-B) + (A + B)           & \text{(\nameref{real:t:props:commutative_add})}  \\
		  & = [(-B) + A] + B           & \text{(\nameref{real:t:props:associative_add})}  \\
		  & = B + [A + (-B)]           & \text{(\nameref{real:t:props:commutative_add})}.
	\end{align*}
	Similarly, by the \nameref{real:t:props:identity_add} we observe that $-A = (-A) + D_0$. Then,
	\begin{align*}
		-A & = (-A) + D_0 = D_0 + (-A) & \text{(\nameref{real:t:props:commutative_add})}    \\
		   & = [B + (-B)] + (-A)       & \text{(\nameref{real:t:props:inverses_add})}       \\
		   & = B + [(-B) + (-A)]       & \text{(\nameref{real:t:props:associative_add})}    \\
		   & = B + [-(B + A)]          & \text{(Part (\ref{real:e:4:3}) of this exercise)}.
	\end{align*}
\end{proof}


%-------------------------------------------------------------------------------------------------------------
\Newpage
\begin{exercise}[Used in Theorem \ref{real:t:props}] %% 1.7.5
	\label{real:e:5}
	Prove Theorem \ref{real:t:props} (\ref{real:t:props:associative_mult}) (\ref{real:t:props:identity_mult}). For this exercise, you may use only Parts (\ref{real:t:props:associative_add}), (\ref{real:t:props:commutative_add}), (\ref{real:t:props:identity_add}), (\ref{real:t:props:inverses_add}), (\ref{real:t:props:trichotomy}), (\ref{real:t:props:transitive}), (\ref{real:t:props:addition_order}) and (\ref{real:t:props:non_triviality}) of the theorem, and anything prior to the theorem.

	\hfill [Use Exercise \ref{real:e:4}.]
\end{exercise}

\begin{proof}
	\hfill

	\PartProof{real:t:props:associative_mult}
	Let $X, Y, Z \in \mathbb{R}$, and suppose that $X \geq D_0$ and $Y \geq D_0$ and $Z \geq D_0$. By the definition of multiplication for the real numbers, it then follows that
	\begin{align*}
		(X Y) Z & = \{ r \in \mathbb{Q} \mid r = (x y)z \text{ for some } x \in X \text{ and } y \in Y \text{ and } z \in Z \}  \\
		        & = \{ r \in \mathbb{Q} \mid r = x (y z) \text{ for some } x \in X \text{ and } y \in Y \text{ and } z \in Z \} \\
		        & = X (Y Z).
	\end{align*}

	Suppose that $X \geq D_0$ and $Y < D_0$. From Exercise \ref{real:e:4} (\ref{real:e:4:2}) it follows that $X (-Y) = -[-X (-Y)] = - X Y$. By Exercise \ref{real:e:4} (\ref{real:e:4:1}) and Exercise \ref{real:e:2} (\ref{real:e:2:1}) we deduce that $X (-Y) = - X Y > D_0$ and $X Y < D_0$. A similar argument shows that if $X < D_0$ and $Y \geq D_0$ then $(-X) Y = -X Y > D_0$ and $X Y < D_0$.

	Now we will prove that $(A B) C = A (B C)$. We consider the following cases.
	\begin{bycases}
		\item $A \geq D_0$ and $B \geq D_0$.
		      \begin{bycases}
			      \item $C \geq D_0$. Clearly, $(A B) C = A (B C)$.
			      \item $C < D_0$. Then $-C < D_0$, and then
			            \begin{align*}
				            (A B) C & = -[(A B) \cdot (-C)] = -[A \cdot (B [-C])] \\
				                    & = -[A \cdot (-B C)] = A (B C).
			            \end{align*}
		      \end{bycases}
		\item \label{real:e:5:c:2} $A < D_0$ and $B \geq D_0$. Then $-A > D_0$ and $A B < D_0$.
		      \begin{bycases}
			      \item $C \geq D_0$. It then follows that
			            \begin{align*}
				            (A B) C & = -[(-A B) \cdot  C] = -[([-A] B) \cdot C] \\
				                    & = -[(-A) \cdot (B C)] = A (B C).
			            \end{align*}
			      \item $C < D_0$. Then $-C > D_0$, and then
			            \begin{align*}
				            (A B) C & = (-A B) \cdot (-C) = ([-A] B) \cdot (-C)             \\
				                    & = (-A) \cdot (B [-C]) = (-A) \cdot (- B C) = A (B C).
			            \end{align*}
		      \end{bycases}
		\item $A \geq D_0$ and $B < D_0$. By a similar argument as Case \ref{real:e:5:c:2}, we deduce that $(A B) C = A (B C)$.
		\item $A < D_0$ and $B < D_0$. It then follows that $-A > D_0$ and $-B > D_0$, and then ${A B = (-A)(-B) > D_0}$.
		      \begin{bycases}
			      \item $C \geq D_0$. Then
			            \begin{align*}
				            (A B) C & = ([-A] [-B]) \cdot C = (-A) \cdot ([-B] C) \\
				                    & = (-A) \cdot (-B C) = A (B C).
			            \end{align*}
			      \item $C < D_0$. Then $-C > D_0$, and then
			            \begin{align*}
				            (A B) C & = -[([-A] [-B]) \cdot (-C)]      \\
				                    & = -[(-A) \cdot ([-B] [-C])]      \\
				                    & = -[(-A) \cdot (B C)] = A (B C).
			            \end{align*}
		      \end{bycases}
	\end{bycases}

	\PartProof{real:t:props:identity_mult}
	We note that $D_1 > D_0$ because of the \nameref{real:t:props:non_triviality}. Suppose that $A \geq D_0$. To see that $A \cdot D_1 \subseteq A$, suppose that $x = a d_1 \in A \cdot D_1$ for some $a \in A$ and $d_1 \in D_1$. From Lemma \ref{cuts:l:diff} (\ref{cuts:l:diff:2}) it follows that $a > 0$, and from the definition of $D_1$ it follows that $d_1 > 1$. Then $a d_1 > a$. Applying Part (\ref{cuts:d:cut:b}) of the definition of \nameref{cuts:d:cuts} to $A$, we deduce that $x = a d_1 \in A$. Hence, $A \cdot D_1 \subseteq A$.

	To see that $A \subseteq A \cdot D_1$, suppose that $x \in A$. By Part (\ref{cuts:d:cut:c}) of the definition of \nameref{cuts:d:cuts} we can find some $y \in A$ such that $y < x$. Let $d = x y^{-1}$. Then $d > 1$, and as a result $d \in D_1$. We then deduce that
	$$
		x = x \cdot 1 = x (y y^{-1}) = (x y) y^{-1} = (y x) y^{-1} = y (x y^{-1}) = y d.
	$$
	Therefore $x \in A \cdot D_1$, or in other words $A \subseteq A \cdot D_1$. Thus, we have shown that if $A \geq D_0$ then $A \cdot D_1 = A$.

	Now suppose that $A < D_0$. By Exercise \ref{real:e:4} (\ref{real:e:4:1}) we see that $-A \geq D_0$. Using Exercise \ref{real:e:4} (\ref{real:e:4:2}) and the case we have already proved, we obtain
	$$
		A \cdot D_1 = - [(-A)D_1] = -(-A) = A.
	$$
\end{proof}


%-------------------------------------------------------------------------------------------------------------
\Newpage
\begin{exercise}[Used in Theorem \ref{real:t:props}] %% 1.7.6
	\label{real:e:6}
	Prove the remaining four cases in the proof of Theorem \ref{real:t:props} (\ref{real:t:props:distributive}).

	\hfill [Use Exercise \ref{real:e:4}.]
\end{exercise}

\begin{proof}
	First, suppose that $A < D_0$ and $B \geq D_0$ and $C \geq D_0$. Then $-A > D_0$, and then
	\begin{align*}
		A(B + C) & = -[(-A)(B + C)] = -[(-A)B + (-A)C] = -[(-[A B]) + (-[A C])] \\
		         & = -[-(A B + A C)] = A B + A C.
	\end{align*}

	Second, suppose that $A < D_0$ and $B < D_0$ and $C \geq D_0$. Then $-A > D_0$ and $-B > D_0$. If $B + C \geq D_0$, then
	\begin{align*}
		A B + A C & = (-A)(-B) + (-[(-A)([-B] + [B + C])])        \\
		          & = (-A)(-B) + (-[(-A)(-B)]) + (-[(-A)(B + C)]) \\
		          & = D_0 + (-[(-A)(B + C)])                      \\
		          & = (-[(-A)(B + C)]) + D_0                      \\
		          & = -[(-A)(B + C)] = A(B + C).
	\end{align*}
	If $B + C < D_0$, then $-(B + C) > D_0$ and
	\begin{align*}
		A B + A C & = (-A)(-B) + (-[(-A)C])               \\
		          & = (-A)([-(B + C)] + C) + (-[(-A)C])   \\
		          & = (-A)[-(B + C)] + (-A)C + (-[(-A)C]) \\
		          & = (-A)[-(B + C)] + D_0                \\
		          & = (-A)[-(B + C)] = A(B + C).
	\end{align*}

	Third, suppose that $A < D_0$ and $B \geq D_0$ and $C < D_0$. This case is just like the previous case, and we omit details.

	Fourth, suppose that $A < D_0$ and $B < D_0$ and $C < D_0$. Then $B + C < D_0$, and $-A > D_0$ and $-B > D_0$ and $-C > D_0$ and $-(B + C) > D_0$. Then,
	\begin{align*}
		A(B + C) & = (-A)[-(B + C)] = (-A)[(-B) + (-C)] \\
		         & = (-A)(-B) + (-A)(-C) = A B + A C.
	\end{align*}
\end{proof}


%-------------------------------------------------------------------------------------------------------------
\Newpage
\begin{exercise}[Used in Theorem \ref{real:t:rat}] %% 1.7.7
	\label{real:e:7}
	Prove Theorem \ref{real:t:rat}.

	\hfill [Use Exercise \ref{real:e:1}.]
\end{exercise}

\begin{proof}
	\hfill

	\PartProof{real:t:rat:1}
	Let $x, y \in \mathbb{Q}$, and suppose that $i(x) = i(y)$. Then $D_x = D_y$. This implies $r > x$ and $r > y$ for all $r \in D_x = D_y$. Suppose to the contrary that $x \neq y$. This means that $x < y$ or $x > y$. If $x < y$ then $y \in D_x$, and as a result $y \in D_y$, which is a contradiction to the definition of $D_y$. Similarly, if $x > y$ then $x \in D_y$, and as a result $x \in D_x$, which is a contradiction to the definition of $D_x$. Hence, $x = y$.

	\PartProof{real:t:rat:2}
	Clearly, by the definition of $i: \mathbb{Q} \to \mathbb{R}$ we have $i(0) = D_0$ and $i(1) = D_1$.

	\PartProof{real:t:rat:3:1}
	We will prove that $D_{r + s} = D_r + D_s$, which will imply that $i(r + s) = i(r) + i(s)$. Suppose that $x \in D_{r + s}$. Then $x > r + s$. Let $a = \frac{x + r - s}{2}$ and $b = \frac{x + s -r}{2}$. We observe that
	$$
		a = \frac{x + r - s}{2} > \frac{(r + s) + r - s}{2} = r
	$$
	and that
	$$
		b = \frac{x + s - r}{2} > \frac{(r + s) + s - r}{2} = s.
	$$
	It then follows that $a \in D_r$ and $b \in D_s$. We deduce that
	$$
		a + b = \frac{x + r - s}{2} + \frac{x + s - r}{2} = x.
	$$
	Hence $x \in D_r + D_s$, and therefore $D_{r + s} \subseteq D_r + D_s$.

	Now suppose that $x \in D_r + D_s$. We can choose some $a \in D_r$ and $b \in D_s$ such that $x = a + b$. Then $a > r$ and $b > s$. We see that $a + b > r + b$ and that $b + r > s + r$. Then $x = a + b > r + s$. Hence $x \in D_{r + s}$, and therefore $D_r + D_s \subseteq D_{r + s}$.

	\PartProof{real:t:rat:3:2}
	By Exercise \ref{real:e:1} (\ref{real:e:1:1}) we know that $D_{-r} = -D_r$, so $i(-r) = -i(r)$, as required.

	\PartProof{real:t:rat:3:3}
	We will prove that $D_{r s} = D_r D_s$, which will imply that $i(r s) = i(r) i(s)$. Suppose that $D_r \geq D_0$ and $D_s \geq D_0$. If $D_s = D_0$ then
	$$
		D_{r s} = D_{r \cdot 0} = D_0 = D_r D_0 = D_r D_s.
	$$
	A similar argument shows that if $D_r = D_0$ then $D_{r s} = D_r D_s$. Next suppose that $D_r > D_0$ and $D_s > D_0$. By Lemma \ref{real:l:mult_req} (\ref{real:l:mult_req:2}) it follows that $r > 0$ and $s > 0$. Suppose further that $x \in D_{r s}$. Then $x > r s$. By Part (\ref{cuts:d:cut:c}) of the definition of \nameref{cuts:d:cuts} applied to $D_{r s}$ we can find some $y \in D_{r s}$ such that $y < x$. We note that
	\begin{align*}
		y > rs & \iff x - (x - y) > rs                        \\
		       & \iff x > rs + (x - y)                        \\
		       & \iff x > r \left( s + (x - y) r^{-1} \right) \\
		       & \iff \frac{x}{s + (x - y) r^{-1}} > r.
	\end{align*}
	By Exercises \ref{rat:e:6} (\ref{rat:e:6:5}) and \ref{rat:e:6} (\ref{rat:e:6:4}) we have $(x - y) r^{-1} > 0$, so $s + (x - y) r^{-1} > s$. Let
	$$
		a = \frac{x}{s + (x - y) r^{-1}} \text{ and } b = s + (x - y) r^{-1}.
	$$
	Then $a \in D_r$ and $b \in D_s$. We also observe that
	$$
		a b = \frac{x}{s + (x - y) r^{-1}} \left( s + (x - y) r^{-1} \right) = x.
	$$
	Hence $x \in D_r + D_s$, and therefore $D_{r s} \subseteq D_r D_s$.

	To see that $D_r D_s \subseteq D_{r s}$, suppose that $x \in D_r D_s$. We can find some $a \in D_r$ and $b \in D_s$ such that ${x = a b}$. Then $a > r$ and $b > s$. It follows from Lemma \ref{real:l:mult_req} (\ref{real:l:mult_req:2}) and Exercise \ref{rat:e:6} (\ref{rat:e:6:7}) that $0 < r s < a b$, and hence $x \in D_{r s}$, which implies $D_r D_s \subseteq D_{r s}$. Thus, we have shown that if $D_r \geq D_0$ and $D_s \geq D_0$ then $D_{r s} = D_r D_s$.

	Now we will prove the remaining three cases. First, suppose that $D_r < D_0$ and ${D_s \geq D_0}$. By Exercise \ref{real:e:4} (\ref{real:e:4:1}) we see that $-D_r > D_0$. Using Exercises \ref{real:e:1} (\ref{real:e:1:1}) and \ref{real:e:4} (\ref{real:e:4:2}) we obtain
	$$
		D_{r s} = -(-D_{r s}) = -D_{-(r s)} = -D_{(-r) s} = -(D_{-r} D_s) = -[(-D_r) D_s] = D_r D_s.
	$$

	Second, suppose that $D_r \geq D_0$ and $D_s < D_0$. This case is just like the previous case, and we omit details.

	Third, suppose that $D_r < D_0$ and $D_s < D_0$. Then $-D_r > D_0$ and $-D_s > D_0$, and we deduce that
	$$
		D_{r s} = D_{-(-r s)} = D_{(-r)(-s)} = D_{-r} D_{-s} = (-D_r)(-D_s) = D_r D_s.
	$$

	\PartProof{real:t:rat:3:4}
	Suppose that $r \neq 0$. By Exercise \ref{real:e:1} (\ref{real:e:1:2}) it follows that $D_{r^{-1}} = [D_r]^{-1}$, so $i(r^{-1}) = [i(r)]^{-1}$, as required.

	\PartProof{real:t:rat:3:5}
	Suppose that $r < s$. Then $s - r > 0$. From Lemma \ref{real:l:mult_req} (\ref{real:l:mult_req:1}) we see that $D_{s - r} > D_0$. Then,
	\begin{align*}
		D_0 < D_{s - r} & \iff D_0 < D_{s + (-r)}                                                                             \\
		                & \iff D_0 < D_s + D_{-r}               & \text{(Part (\ref{real:t:rat:3:1}) of this exercise)}       \\
		                & \iff D_0 < D_s + (-D_r)               & \text{(Part (\ref{real:e:1:1}) of Exercise \ref{real:e:1})} \\
		                & \iff D_0 + D_r < [D_s + (-D_r)] + D_r & \text{(\nameref{real:t:props:addition_order})}              \\
		                & \iff D_0 + D_r < D_s + [(-D_r) + D_r] & \text{(\nameref{real:t:props:associative_add})}             \\
		                & \iff D_r + D_0 < D_s + [D_r + (-D_r)] & \text{(\nameref{real:t:props:commutative_add})}             \\
		                & \iff D_r + D_0 < D_s + D_0            & \text{(\nameref{real:t:props:inverses_add})}                \\
		                & \iff D_r < D_s                        & \text{(\nameref{real:t:props:identity_add})}.
	\end{align*}
\end{proof}
