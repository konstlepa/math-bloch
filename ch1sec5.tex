%=============================================================================================================
%= SECTION 1.5
%=============================================================================================================
\section{Constructing the Rational Numbers}
\label{rat}

\begin{definition} %% 1.5.1
	Let $\mathbb{Z}^{*} = \mathbb{Z} - \{0\}$. The relation $\asymp$ on $\mathbb{Z} \times \mathbb{Z}^{*}$ is defined by $(x, y) \asymp (z, w)$ if and only if $x w = y z$, for all $(x, y), (z, w) \in \mathbb{Z} \times \mathbb{Z}^{*}$.
\end{definition}

\begin{lemma} %% 1.5.2
	\label{rat:l:equiv}
	The relation $\asymp$ is an equivalence relation.
\end{lemma}

\begin{definition} %% 1.5.3
	\label{rat:d:rat}
	The set of \emph{rational numbers}, denoted $\mathbb{Q}$, is the set of equivalence classes of $\mathbb{Z} \times \mathbb{Z}^{*}$ with respect to the equivalence relation $\asymp$.

	The elements $\overline{0}, \overline{1} \in \mathbb{Q}$ are defined by $\overline{0} = [(0, 1)]$ and $\overline{1} = [(1, 1)] .$ Let $\mathbb{Q}^{*} = \mathbb{Q} - \{\overline{0}\}$. The binary operations $+$ and $\cdot$ on $\mathbb{Q}$ are defined by
	\begin{align*}
		[(x, y)] + [(z, w)]     & = [(x w + y z, y w)] \\
		[(x, y)] \cdot [(z, w)] & = [(x z, y w)]
	\end{align*}
	for all $[(x, y)], [(z, w)] \in \mathbb{Q}$. The unary operation $-$ on $\mathbb{Q}$ is defined by $-[(x, y)] = [(-x, y)]$ for all $[(x, y)] \in \mathbb{Q}$. The unary operation $^{-1}$ on $\mathbb{Q}^{*}$ is defined by $[(x, y)]^{-1} = [(y, x)]$ for all $[(x, y)] \in \mathbb{Q}^{*}$. The relation $<$ on $\mathbb{Q}$ is defined by $[(x, y)] < [(z, w)]$ if and only if either $x w < y z$ when $y>0$ and $w > 0$ or when $y < 0$ and $w < 0$, and $x w > y z$ when $y > 0$ and $w < 0$ or when $y < 0$ and $w > 0$, for all $[(x, y)], [(z, w)] \in \mathbb{Q}$. The relation $\leq$ on $\mathbb{Q}$ is defined by $[(x, y)] \leq [(z, w)]$ if and only if $[(x, y)] < [(z, w)]$ or $[(x, y)] = [(z, w)]$, for all $[(x, y)], [(z, w)] \in \mathbb{Q}$.
\end{definition}

\begin{lemma} %% 1.5.4
	\label{rat:l:ops}
	The binary operations $+$ and $\cdot$, the unary operations $-$ and $^{-1}$, and the relation $<$, all on $\mathbb{Q}$, are well-defined.
\end{lemma}

\begin{theorem} %% 1.5.5
	\label{rat:t:props}
	Let $r, s, t \in \mathbb{Q}$.
	\begin{enumerate}
		\item $(r + s) + t = r + (s + t)$ \quad (\nameref*{rat:t:props:associative_add}). \xlabel[Associative Law for Addition]{rat:t:props:associative_add}
		\item $r + s = s + r$ \quad (\nameref*{rat:t:props:commutative_add}). \xlabel[Commutative Law for Addition]{rat:t:props:commutative_add}
		\item $r + \overline{0} = r$ \quad (\nameref*{rat:t:props:identity_add}). \xlabel[Identity Law for Addition]{rat:t:props:identity_add}
		\item $r + (-r) = \overline{0}$ \quad (\nameref*{rat:t:props:inverses_add}). \xlabel[Inverses Law for Addition]{rat:t:props:inverses_add}
		\item $(r s) t = r(s t)$ \quad (\nameref*{rat:t:props:associative_mult}). \xlabel[Associative Law for Multiplication]{rat:t:props:associative_mult}
		\item $r s = s r$ \quad (\nameref*{rat:t:props:commutative_mult}). \xlabel[Commutative Law for Multiplication]{rat:t:props:commutative_mult}
		\item $r \cdot \overline{1} = r$ \quad (\nameref*{rat:t:props:identity_mult}). \xlabel[Identity Law for Multiplication]{rat:t:props:identity_mult}
		\item If $r \neq \overline{0}$, then $r \cdot r^{-1} = \overline{1}$ \quad (\nameref*{rat:t:props:inverses_mult}). \xlabel[Inverses Law for Multiplication]{rat:t:props:inverses_mult}
		\item $r(s + t) = r s + r t$ \quad (\nameref*{rat:t:props:distributive}). \xlabel[Distributive Law]{rat:t:props:distributive}
		\item Precisely one of $r < s$ or $r = s$ or $r > s$ holds \quad (\nameref*{rat:t:props:trichotomy}). \xlabel[Trichotomy Law]{rat:t:props:trichotomy}
		\item If $r < s$ and $s < t$, then $r < t$ \quad (\nameref*{rat:t:props:transitive}). \xlabel[Transitive Law]{rat:t:props:transitive}
		\item If $r < s$ then $r + t < s + t$ \quad (\nameref*{rat:t:props:addition_order}). \xlabel[Addition Law for Order]{rat:t:props:addition_order}
		\item If $r < s$ and $t > \overline{0}$, then $r t < s t$ \quad (\nameref*{rat:t:props:multiplication_order}). \xlabel[Multiplication Law for Order]{rat:t:props:multiplication_order}
		\item $\overline{0} \neq \overline{1}$ \quad (\nameref*{rat:t:props:non_triviality}). \xlabel[Non-Triviality]{rat:t:props:non_triviality}
	\end{enumerate}
\end{theorem}

\begin{theorem} %% 1.5.6
	\label{rat:t:int}
	Let $i: \mathbb{Z} \to \mathbb{Q}$ be defined by $i(x) = [(x, 1)]$ for all $x \in \mathbb{Z}$.
	\begin{enumerate}
		\item \label{rat:t:int:injective}
		      The function $i: \mathbb{Z} \to \mathbb{Q}$ is injective.
		\item \label{rat:t:int:zero_one}
		      $i(0) = \overline{0}$ and $i(1) = \overline{1}$.
		\item \label{rat:t:int:props}
		      Let $x, y \in \mathbb{Z}$. Then
		      \begin{enumerate}
			      \item \label{rat:t:int:props:add}
			            $i(x + y) = i(x) + i(y)$;
			      \item \label{rat:t:int:props:neg}
			            $i(-x) = -i(x)$;
			      \item \label{rat:t:int:props:mult}
			            $i(x y) = i(x) i(y)$;
			      \item \label{rat:t:int:props:less}
			            $x < y$ if and only if $i(x) < i(y)$.
		      \end{enumerate}
		\item \label{rat:t:int:div}
		      For each $r \in \mathbb{Q}$ there are $x, y \in \mathbb{Z}$ such that $y \neq 0$ and $r = i(x)(i(y))^{-1}$.
	\end{enumerate}
\end{theorem}

\begin{definition} %% 1.5.7
	\label{rat:d:div}
	The binary operation $-$ on $\mathbb{Q}$ is defined by $r - s = r + (-s)$ for all $r, s \in \mathbb{Q}$. The binary operation $\div: \mathbb{Q} \times \mathbb{Q}^{*} \to \mathbb{Q}$ is defined by $r \div s = r s^{-1}$ for all $(r, s) \in \mathbb{Q} \times \mathbb{Q}^{*}$; we also let $0 \div s = 0 \cdot s^{-1} = 0$ for all $s \in \mathbb{Q}^{*}$. The number $r \div s$ is also denoted $\frac{r}{s}$.
\end{definition}

\begin{lemma} %% 1.5.8
	\label{rat:l:alt_def}
	Let $a, c \in \mathbb{Z}$ and $b, d \in \mathbb{Z}^{*}$.
	\begin{enumerate}
		\item \label{rat:l:alt_def:1}
		      $\frac{a}{b} = \frac{c}{d}$ if and only if $a d = b c$.
		\item \label{rat:l:alt_def:2}
		      $\frac{a}{b} + \frac{c}{d} = \frac{a d + b c}{b d}$.
		\item \label{rat:l:alt_def:3}
		      $-\frac{a}{b} = \frac{-a}{b}$.
		\item \label{rat:l:alt_def:4}
		      $\frac{a}{b} \cdot \frac{c}{d} = \frac{a c}{b d}$.
		\item \label{rat:l:alt_def:5}
		      If $a \neq 0$, then $\left( \frac{a}{b} \right)^{-1} = \frac{b}{a}$.
		\item \label{rat:l:alt_def:6}
		      If $b > 0$ and $d > 0$, or if $b < 0$ and $d < 0$, then $\frac{a}{b} < \frac{c}{d}$ if and only if $a d < b c$; if $b > 0$ and $d < 0$, or if $b < 0$ and $d > 0$, then $\frac{a}{b} < \frac{c}{d}$ if and only if $a d > b c$.
	\end{enumerate}
\end{lemma}


%-------------------------------------------------------------------------------------------------------------
\Newpage
\begin{restatable}[Not in the book]{lemma}{ratLessRelL} %% 1.5.9
	\label{rat:l:less_relation}
	Let
	$$
		P = \{ (x, y) \in \mathbb{Q} \mid \text{there is either $x > 0$ and $y > 0$ or $x < 0$ and $y < 0$} \}.
	$$
	Let $r, s \in \mathbb{Q}$.
	\begin{enumerate}
		\item \label{rat:l:less_relation:1}
		      $r < s$ if and only if $(-r) + s \in P$.
		\item \label{rat:l:less_relation:2}
		      if $r, s \in P$, then $r + s \in P$ and $r s \in P$.
		\item \label{rat:l:less_relation:3}
		      if $r \in \mathbb{Q}^{*}$, then precisely one of $r \in P$ or $-r \in P$ holds.
	\end{enumerate}
\end{restatable}

\begin{proof}
	Suppose that $r = [(x, y)]$ and that $s = [(z, w)]$.

	\PartProof{rat:l:less_relation:1}
	Suppose that $r < s$. Then $[(x, y)] < [(z, w)]$. We consider the following two cases.
	\begin{bycases}
		\item There is either $y > 0$ and $w > 0$ or $y < 0$ and $w < 0$. Without loss of generality, suppose that $y > 0$ and $w > 0$. Then $x w < y z$. Adding $-x w$ to both sides of this inequality we obtain
		      $$
			      y z + (-x w) > x w + (-x w) = 0,
		      $$
		      so $(-x w) + y z > 0$. Also, $y > 0$ and $w > 0$ imply $y w > 0$. Hence,
		      $$
			      [([-x w] + y z, y w)] = (-r) + s \in P.
		      $$
		\item There is either $y < 0$ and $w > 0$ or $y > 0$ and $w < 0$. A similar argument shows that $(-x w) + yz < 0$ and $y w < 0$, and hence
		      $$
			      [([-x w] + y z, y w)] = (-r) + s \in P.
		      $$
	\end{bycases}

	Now suppose that $(-r) + s \in P$, which means that $[([-x w] + y z, y w)] \in P$. From the definition of $P$ this implies the following two cases.
	\begin{bycases}
		\item $(-x w) + y z > 0$ and $y w > 0$. Adding $x w$ to the inequality $(-x w) + y z > 0$ and doing some rearranging we obtain $x w < y z$. Next suppose to the contrary that either $y > 0$ and $w < 0$ or $y < 0$ and $w > 0$, or in other words there is either $y > 0$ and $-w > 0$ or $-y > 0$ and $w > 0$. Then
		      $$
			      y(-w) = (-y)w = -yw > 0,
		      $$
		      so $y w < 0$, which is a contradiction to hypothesis on $y w$. Hence either $y > 0$ and $w > 0$ or $y < 0$ and $w < 0$. By Definition \ref{rat:d:rat} we can conclude that $[(x, y)] = r < s = [(z, w)]$.
		\item $(-x w) + y z < 0$ and $y w < 0$. A similar argument shows that $r < s$.
	\end{bycases}

	\PartProof{rat:l:less_relation:2}
	Suppose that $r, s \in P$, which means that $[(x, y)] \in P$ and $[(z, w)] \in P$. We consider the following four cases.
	\begin{bycases}
		\item $x > 0$, $y > 0$, $z > 0$, and $w > 0$. Then $x w > 0$, $y z > 0$, $x z > 0$, and $y w > 0$, and as a result $x w + y z > 0$. Hence, $r + s = [(x w + y z, y w)] \in P$ and $rs = [(x z, y w)] \in P$.
		\item $x < 0$, $y < 0$, $z < 0$, and $w < 0$. Then $-x > 0$, $-y > 0$, $-z > 0$, and $-w > 0$. Because $(-a)(-b) = a b$ for all $a, b \in \mathbb{Z}$ this case is similar to the previous case. Hence, $r + s \in P$ and $rs \in P$.
		\item $x > 0$, $y > 0$, $z < 0$, and $w < 0$. Then $x w < 0$, $y z < 0$, $x z < 0$ and $y w < 0$. It then follows that $x w + y z < 0$. Hence, $r + s = [(x w + y z, y w)] \in P$ and $rs = [(x z, y w)] \in P$.
		\item $x < 0$, $y < 0$, $z > 0$, and $w > 0$. A similar argument shows that $r + s \in P$ and $rs \in P$.
	\end{bycases}

	\PartProof{rat:l:less_relation:3}
	Suppose that $r \in \mathbb{Q}^{*}$, or in other words $r \in \mathbb{Q}$ and $r \not= 0$. Suppose that $r \notin P$. Then either $x > 0$ and $y < 0$ or $x < 0$ and $y > 0$. It then follows either $-x < 0$ and $y < 0$ or $-x > 0$ and $y > 0$. Hence, $[(-x, y)] = -r \in P$.

	Now suppose to the contrary that $r \in P$ and $-r \in P$. Because $r \in P$ there is either $x > 0$ and $y > 0$ or $x < 0$ and $y < 0$. Without loss of generality, suppose that $x > 0$ and $y > 0$. Because $-r \in P$ we consider the following two cases.
	\begin{bycases}
		\item $-x > 0$ and $y > 0$. Then $-x > 0$ implies $x < 0$, which is a contradiction to the fact that $x > 0$. Hence, $-r \notin P$.
		\item $-x < 0$ and $y < 0$. Then there is a contradiction to the fact that $y > 0$, and hence $-r \notin P$.
	\end{bycases}

	Thus, we can conclude that precisely one of $r \in P$ or $-r \in P$ holds.
\end{proof}


%-------------------------------------------------------------------------------------------------------------
\Newpage
\begin{restatable}[Not in the book]{lemma}{ratNegationL} %% 1.5.9
	\label{rat:l:negation}
	Let $r, s \in \mathbb{Q}$. Then $(-r) + (-s) = -(r + s)$.
\end{restatable}

\begin{proof}
	By the \nameref{rat:t:props:inverses_add} we know that $(r + s) + [-(r + s)] = \overline{0}$. Adding $(-r) + (-s)$ to both sides of this equation we obtain
	$$
		(r + s) + [-(r + s)] + (-r) + (-s) = (-r) + (-s).
	$$
	Then by repeated use of the \hyperref[rat:t:props:associative_add]{Associative}, \hyperref[rat:t:props:commutative_add]{Commutative}, \hyperref[rat:t:props:inverses_add]{Inverses}, and \hyperref[rat:t:props:identity_add]{Identity} Laws for Addition we can conclude that $-(r + s) = (-r) + (-s)$.
\end{proof}


%-------------------------------------------------------------------------------------------------------------
\Newpage
\begin{exercise}[Used in Lemma \ref{rat:l:equiv}] %% 1.5.1
	Complete the proof of Lemma \ref{rat:l:equiv}. That is, prove that the relation $\asymp$ is reflexive and symmetric.
\end{exercise}

\begin{proof}
	\begin{notmine}
		Let $(x, y), (z, w), (u, v) \in \mathbb{Z} \times \mathbb{Z}^{*}$. Suppose that $(x, y) \asymp (z, w)$ and $(z, w) \asymp (u, v)$. Then $x w = y z$ and $z v = w u$. It follows that $(x w)v = (y z) v$ and $y(z v) = y(w u)$, which implies that $(x v) w = (y z) v$ and $(y z) v = (y u) w$, and hence $(x v) w = (y u) w$. We know that $w \neq 0$, and therefore we deduce that $x v = y u$. It follows that $(x, y) \asymp (u, v)$. Therefore $\asymp$ is transitive.
	\end{notmine}
	Since $x y = y x$, it follows that $(x, y) \asymp (x, y)$, and hence $\asymp$ is reflexive. Now to see that $\asymp$ is symmetric, suppose that $(x, y) \asymp (z, w)$. Then $x w = y z$. This implies that $y z = x w$, which implies that $z y = w x$. Hence, $(z, w) \asymp (x, y)$.
\end{proof}


%-------------------------------------------------------------------------------------------------------------
\Newpage
\ratLessRelL*

\begin{exercise}[Used in Lemma \ref{rat:l:ops}] %% 1.5.2
	Complete the proof of Lemma \ref{rat:l:ops}. That is, prove that the binary operation $+$, the unary operation $^{-1}$ and the relation $<$, all on $\mathbb{Q}$, are well-defined.
\end{exercise}

\begin{proof}
	\begin{notmine}
		Let $(x, y), (z, w), (a, b), (c, d) \in \mathbb{Z} \times \mathbb{Z}^{*}$. Suppose that $[(x, y)] = [(a, b)]$ and that $[(z, w)] = [(c, d)]$.

		By hypothesis we know that $(x, y) \asymp (a, b)$ and $(z, w) \asymp (c, d)$. Hence $x b = y a$ and $z d = w c$. By multiplying these two equations and doing some rearranging we obtain $(x z)(b d) = (y w)(a c)$, and this implies that $[(x z, y w)] = [(a c, b d)]$. Therefore $\cdot$ is well-defined. Also, from $x b = y a$ we deduce that $(-x)b = y(-a)$, and hence $[(-x, y)] = [(-a, b)]$. Therefore $-$ is well-defined.
	\end{notmine}

	Multiplying the equation $x b = y a$ by $w d$ we get $(x b)(w d) = (y a)(w d)$, and multiplying the equation $z d = w c$ by $y b$ we get $(z d)(y b) = (w c)(y b)$. By adding these two equations we obtain $(x b)(w d) + (z d)(y b) = (y a)(w d) + (w c)(y b)$, and doing some rearranging we have $(x w + y z)(b d) = (a d + b c)(y w)$, which implies that $[(x w + yz, y w)] = [(a d + b c, b d)]$. Therefore $+$ is well-defined. From $x b = y a$ it follows that $y a = x b$, so $[(y, x)] = [(b, a)]$. Hence, $^{-1}$ is well-defined.

	To see that $<$ is well-defined, suppose that $[(x, y)] < [(z, w)]$. From Part (\ref{rat:l:less_relation:2}) of Lemma \ref{rat:l:less_relation} it follows that $[(-x, y)] + [(z, w)] \in P$, so $[([-x w] + y z, y w)] \in P$. Then there is either $(-x w) + y z > 0$ and $y w > 0$ or $(-x w) + y z < 0$ and $y w < 0$. Without loss of generality, suppose that $(-x w) + y z > 0$ and $y w > 0$. Adding $x w$ to both sides of the inequality $(-x w) + y z > 0$ we get
	$$
		x w + (-x w) + y z = 0 + y z > x w + 0,
	$$
	and as a result ${x w < y z}$. We know that $b d = y w > 0$, so multiplying both sides of the inequality ${x w < y z}$ by $b d$ and doing some rearranging we obtain ${(x b)(w d) < (z d)(y b)}$. Because $x b = y a$ and $z d = w c$ we have ${(y a)(w d) < (w c)(y b)}$. Doing some rearranging again we get $(a d)(y w) < (b c)(y w)$. We have $y \not= 0$ and $w \not= 0$, so canceling yields $a d < b c$, and hence $[(a, b)] < [(c, d)]$.
\end{proof}


%-------------------------------------------------------------------------------------------------------------
\Newpage
\ratLessRelL*

\begin{exercise}[Used in Theorem \ref{rat:t:props} and Theorem \ref{rat:t:int}] %% 1.5.3
	\label{rat:e:3}
	Let $x \in \mathbb{Z}$ and $y \in \mathbb{Z}^{*}$.
	\begin{enumerate}
		\item \label{rat:e:3:1}
		      Prove that $[(x, y)] = \overline{0}$ if and only if $x = 0$.
		\item \label{rat:e:3:2}
		      Prove that $[(x, y)] = \overline{1}$ if and only if $x = y$.
		\item \label{rat:e:3:3}
		      Prove that $\overline{0} < [(x, y)]$ if and only if $0 < x y$.
	\end{enumerate}
\end{exercise}

\begin{proof}[(\ref{rat:e:3:1})]
	Suppose that $[(x, y)] = \overline{0}$. Then,
	\begin{align*}
		[(x, y)] = \overline{0} & \iff [(x, y)] = [(0, 1)]   \\
		                        & \iff (x, y) \asymp (0, 1)  \\
		                        & \iff x \cdot 1 = y \cdot 0 \\
		                        & \iff x = 0.
	\end{align*}
\end{proof}

\begin{proof}[(\ref{rat:e:3:2})]
	Suppose that $[(x, y)] = \overline{1}$. Then,
	\begin{align*}
		[(x, y)] = \overline{1} & \iff [(x, y)] = [(1, 1)]   \\
		                        & \iff (x, y) \asymp (1, 1)  \\
		                        & \iff x \cdot 1 = y \cdot 1 \\
		                        & \iff x = y.
	\end{align*}
\end{proof}

\begin{proof}[(\ref{rat:e:3:3})]
	Suppose that $\overline{0} < [(x, y)]$. Using Part (\ref{rat:l:less_relation:1}) of Lemma \ref{rat:l:less_relation}  we have
	\begin{align*}
		\overline{0} < [(x, y)] & \iff [(0, 1)] < [(x, y)]                         \\
		                        & \iff (-[(0, 1)]) + [(x, y)] \in P                \\
		                        & \iff [(-0, 1)] + [(x, y)] \in P                  \\
		                        & \iff [(0, 1)] + [(x, y)] \in P                   \\
		                        & \iff [(0 \cdot y + x \cdot 1, 1 \cdot y )] \in P \\
		                        & \iff [(y \cdot 0 + x \cdot 1, y \cdot 1)] \in P  \\
		                        & \iff [(0 + x, y)] \in P                          \\
		                        & \iff [(x, y)] \in P.
	\end{align*}
	This implies that there is either $x > 0$ and $y > 0$ or $x < 0$ and $y < 0$. If $x > 0$ and $y > 0$ then clearly $0 < x y$. Now suppose that $x < 0$ and $y < 0$. Then $-x > 0$ and $-y > 0$, and we can conclude that $0 < (-x)(-y) = -(-x y) = x y$. This process can be done backwards, and hence $\overline{0} < [(x, y)]$ if and only if $0 < x y$.
\end{proof}


%-------------------------------------------------------------------------------------------------------------
\Newpage
\ratLessRelL*

\ratNegationL*

\begin{exercise}[Used in Theorem \ref{rat:t:props}] %% 1.5.4
	Prove Theorem \ref{rat:t:props} (\ref{rat:t:props:associative_add}) (\ref{rat:t:props:commutative_add}) (\ref{rat:t:props:identity_add}) (\ref{rat:t:props:associative_mult}) (\ref{rat:t:props:commutative_mult}) (\ref{rat:t:props:inverses_mult}) (\ref{rat:t:props:distributive}) (\ref{rat:t:props:transitive}) (\ref{rat:t:props:addition_order}) (\ref{rat:t:props:non_triviality}).
\end{exercise}

\begin{proof}
	Suppose that $r = [(x, y)]$, that $s = [(z, w)]$ and that $t = [(u, v)]$ for some $x, z, u \in \mathbb{Z}$ and $y, w, v \in \mathbb{Z}^{*}$.

	\PartProof{rat:t:props:associative_add}
	We have
	\begin{align*}
		(r + s) + t & = ([(x, y)] + [(z, w)]) + [(u, v)]     \\
		            & = [(x w + y z, y w)] + [(u, v)]        \\
		            & = [((x w + y z)v + (y w)u, (y w)v)]    \\
		            & = [((x w)v + (y z)v + (y w)u, (y w)v)  \\
		            & = [(x(w v) + y(z v) + y(w u), y(w v))] \\
		            & = [(x(w v) + y(z v + w u), y(w v))]    \\
		            & = [(x, y)] + [(z v + w u, w v)]        \\
		            & = [(x, y)] + ([(z, w)] + [(u, v)])     \\
		            & = r + (s + t).
	\end{align*}

	\PartProof{rat:t:props:commutative_add}
	We have
	\begin{align*}
		r + s & = [(x, y)] + [(z, w)]                     \\
		      & = [(x w + y z, y w)] = [(y z + x w, y w)] \\
		      & = [(z y + w x, w y)] = s + r.
	\end{align*}

	\PartProof{rat:t:props:identity_add}
	We have
	\begin{align*}
		r + \overline{0} & = [(x, y)] + [(0, 1)]                                 \\
		                 & = [(x \cdot 1 + y \cdot 0, y \cdot 1)] = [(x + 0, y)] \\
		                 & = [(x, y)] = r.
	\end{align*}

	\PartProof{rat:t:props:associative_mult}
	We have
	\begin{align*}
		(r s)t & = ([(x, y)] \cdot [(z, w)]) \cdot [(u, v)]         \\
		       & = [(x z, y w)] \cdot [(u, v)] = [((x z)u, (y w)v)] \\
		       & = [(x(z u), y(w v))] = [(x, y)] \cdot [(z u, w v)] \\
		       & = [(x, y)] \cdot ([z, w] \cdot [(u, v)]) = r(s t).
	\end{align*}

	\PartProof{rat:t:props:commutative_mult}
	We have
	\begin{align*}
		r s & = [(x, y)] \cdot [(z, w)]         \\
		    & = [(x z, y w)] \cdot [(z x, w y)] \\
		    & = [(z, w)] \cdot [(x, y)] = s r.
	\end{align*}

	\PartProof{rat:t:props:inverses_mult}
	Suppose that $r \not= \overline{0}$, which means that $r \in \mathbb{Q}^{*}$. Then,
	\begin{align*}
		r \cdot r^{-1} & = [(x, y)] \cdot [(x, y)]^{-1} = [(x, y)] \cdot [(y, x)] \\
		               & = [(x y, y x)] = [(x y, x y)] = \overline{1},
	\end{align*}
	where the last equality holds by Exercise \ref{rat:e:3} (\ref{rat:e:3:2}).

	\PartProof{rat:t:props:distributive}
	Using Exercise \ref{rat:e:3} (\ref{rat:e:3:2}) we note that $[(y, y)] = \overline{1}$. By Part (\ref{rat:t:props:identity_mult}) of this theorem we know that $r(s + t) = r(s + t) \cdot \overline{1}$. Then,
	\begin{align*}
		r(s + t) & = r(s + t) \cdot \overline{1}                         \\
		         & = [(x, y)] \cdot ([(z, w)] + [(u, v)]) \cdot [(y, y)] \\
		         & = [(x, y)] \cdot [(z v + w u, w v)] \cdot [(y, y)]    \\
		         & = [(x(z v + w u), y(w v))] \cdot [(y, y)]             \\
		         & = [(x(z v + w u)y, y(w v)y)]                          \\
		         & = [([x(z v) + x(w u)]y, y (w v) y)]                   \\
		         & = [(x(z v)y + x(w u)y, y (w v) y)]                    \\
		         & = [(x z v y + x w u y, y w v y)]                      \\
		         & = [(x z y v + y w x u, y w y v)]                      \\
		         & = [([x z] [y v] + [y w] [x u], [y w] [y v])]          \\
		         & = [(x z, y w)] + [(x u, y v)]                         \\
		         & = rs + rt.
	\end{align*}

	\PartProof{rat:t:props:transitive}
	Suppose that $r < s$ and $s < t$. According to Part (\ref{rat:l:less_relation:1}) of Lemma \ref{rat:l:less_relation} we see that $(-r) + s \in P$ and $(-s) + t \in P$, so using Part (\ref{rat:l:less_relation:2}) of Lemma \ref{rat:l:less_relation} we deduce that $(-r) + s + (-s) + t \in P$. Then by repeated use of the \hyperref[rat:t:props:associative_add]{Associative}, \hyperref[rat:t:props:inverses_add]{Inverses}, and \hyperref[rat:t:props:identity_add]{Identity} Laws for Addition we obtain $(-r) + t \in P$. Hence, $r < t$.

	\PartProof{rat:t:props:addition_order}
	Suppose that $r < s$. By Part (\ref{rat:l:less_relation:1}) of Lemma \ref{rat:l:less_relation} we know that $(-r) + s \in P$, and using the \hyperref[rat:t:props:identity_add]{Identity} and \hyperref[rat:t:props:inverses_add]{Inverses} Laws for Addition we observe that
	$$
		(-r) + s = (-r) + s + \overline{0} = (-r) + s + t + (-t) \in P.
	$$
	By repeated use of the \hyperref[rat:t:props:associative_add]{Associative} and \hyperref[rat:t:props:commutative_add]{Commutative} for Addition we obtain
	$$
		([-r] + [-t]) + (s + t) \in P.
	$$
	By Lemma \ref{rat:l:negation} we see that $(-r) + (-t) = -(r + t)$. Then $[-(r + t)] + (s + t) \in P$, and hence $r + t < s + t$.

	\PartProof{rat:t:props:non_triviality}
	Suppose to the contrary that $\overline{0} = \overline{1}$. Then $[(0, 1)] = [(1, 1)]$, which means that $(0, 1) \asymp (1, 1)$. But then $0 \cdot 1 = 1 \cdot 1$, and by the \nameref{rat:t:props:identity_mult} it follows that $0 = 1$, which contradicts \nameref{int2:d:oid:non_triviality} for integer numbers. Hence, $\overline{0} \not= \overline{1}$.
\end{proof}


%-------------------------------------------------------------------------------------------------------------
\Newpage
\begin{exercise}[Used in Theorem \ref{rat:t:int}] %% 1.5.5
	Prove Theorem \ref{rat:t:int} (\ref{rat:t:int:injective}) (\ref{rat:t:int:zero_one}) (\ref{rat:t:int:props}).
\end{exercise}

\begin{proof}
	\hfill

	\PartProof{rat:t:int:injective}
	Let $a, b \in \mathbb{Z}$, and suppose that $i(a) = i(b)$. From the definition of $i: \mathbb{Z} \to \mathbb{Q}$, it follows that $[(a, 1)] = [(b, 1)]$, which means that $(a, 1) \asymp (b, 1)$. Then $a \cdot 1 = 1 \cdot b$, and hence $a = b$.

	\PartProof{rat:t:int:zero_one}
	By the definition of $i: \mathbb{Z} \to \mathbb{Q}$ we have $i(0) = [(0, 1)]$ and $i(1) = [(1, 1)]$, and from Definition \ref{rat:d:rat} we obtain $i(0) = \overline{0}$ and $i(1) = \overline{1}$, as required.

	\PartProof{rat:t:int:props:add}
	We have
	\begin{align*}
		i(x + y) & = [(x + y, 1)] = [(x \cdot 1 + 1 \cdot y, 1 \cdot 1)] \\
		         & = [(x, 1)] + [(y, 1)] = i(x) + i(y).
	\end{align*}

	\PartProof{rat:t:int:props:neg}
	We have
	$$
		i(-x) = [(-x, 1)] = -[(x, 1)] = -i(x).
	$$

	\PartProof{rat:t:int:props:mult}
	We have
	$$
		i(xy) = [(xy, 1)] = [(xy, 1 \cdot 1)] = [(x, 1)] \cdot [(y, 1)] = i(x)i(y).
	$$

	\PartProof{rat:t:int:props:less}
	Suppose that $x < y$. From Lemma \ref{int2:l:props} (\ref{int2:l:props:9}) we know that $1 > 0$, so ${x \cdot 1 < y \cdot 1}$. Then $x \cdot 1 < 1 \cdot y$, which means that $[(x, 1)] < [(y, 1)]$, or in other words $i(x) < i(y)$. This process can be done backwards, and hence $x < y$ if and only if $i(x) < i(y)$.
\end{proof}


%-------------------------------------------------------------------------------------------------------------
\Newpage

\begin{exercise}[Used in Exercise \ref{cuts:e:2}, Theorem \ref{real:t:props} and Exercise \ref{real:e:3}] %% 1.5.6
	\label{rat:e:6}
	Let ${r, s, p, q \in \mathbb{Q}}$.
	\begin{enumerate}
		\item Prove that $-1 < 0 < 1$. \label{rat:e:6:1}
		\item Prove that if $r < s$ then $-s < -r$. \label{rat:e:6:2}
		\item Prove that $r \cdot 0 = 0$. \label{rat:e:6:3}
		\item Prove that if $r > 0$ and $s > 0$, then $r + s > 0$ and $r s > 0$. \label{rat:e:6:4}
		\item Prove that if $r > 0$, then $\frac{1}{r} > 0$. \label{rat:e:6:5}
		\item Prove that if $0 < r < s$, then $\frac{1}{s} < \frac{1}{r}$. \label{rat:e:6:6}
		\item Prove that if $0 < r < p$ and $0 < s < q$, then $r s < p q$. \label{rat:e:6:7}
	\end{enumerate}
\end{exercise}

\begin{proof}[(\ref{rat:e:6:1})]
	From Lemma \ref{int2:l:props} (\ref{int2:l:props:9}) we know that $0 < 1$ for $0,1 \in \mathbb{Z}$. Adding $-1$ to this inequality we obtain $-1 < 1 + (-1) = 0$, so $-1 < 0 < 1$ for $-1,0,1 \in \mathbb{Z}$. By Theorem\,\ref{rat:t:int} (\ref{rat:t:int:props:less}) it follows that $i(-1) < i(0) < i(1)$. Because of Theorem \ref{rat:t:int} (\ref{rat:t:int:props:neg}) we see that $i(-1) = -i(1)$, and hence by Theorem \ref{rat:t:int} (\ref{rat:t:int:zero_one}) we deduce that $-1 < 0 < 1$.
\end{proof}

\begin{proof}[(\ref{rat:e:6:2})]
	Suppose that $r < s$. By the \nameref{rat:t:props:addition_order} we obtain
	$$
		r + [(-r) + (-s)] < s + [(-r) + (-s)].
	$$
	By repeated use of the \hyperref[rat:t:props:associative_add]{Associative} and \hyperref[rat:t:props:commutative_add]{Commutative} Laws for Addition we deduce that
	$$
		(-s) + [r + (-r)] < (-r) + [s + (-s)].
	$$
	Because of the \nameref{rat:t:props:inverses_add} we see that $r + (-r) = 0$ and $s + (-s) = 0$, so $(-s) + 0 < (-r) + 0$, and hence by the \nameref{rat:t:props:identity_add} we can conclude that $-s < -r$.
\end{proof}

\begin{proof}[(\ref{rat:e:6:3})]
	Let $r = \frac{x}{y}$ for $x,y \in \mathbb{Z}$. Then $r \cdot 0 = \frac{x}{y} \cdot \frac{0}{1}$. By Lemma \ref{rat:l:alt_def} (\ref{rat:l:alt_def:4}) we obtain
	$$
		\frac{x}{y} \cdot \frac{0}{1} = \frac{x \cdot 0}{y \cdot 1} = \frac{0}{y},
	$$
	and from Definition \ref{rat:d:div} we see that $\frac{0}{y} = 0$. Hence, $r \cdot 0 = 0$.

\end{proof}

\begin{proof}[(\ref{rat:e:6:4})]
	Suppose that $r > 0$ and $s > 0$. By the \nameref{rat:t:props:addition_order} we obtain $r + s > 0 + s$. Using the \hyperref[rat:t:props:commutative_add]{Commutative} and \hyperref[rat:t:props:identity_add]{Identity} we observe that $0 + s = s + 0 = s$, so $r + s > s$. By hypothesis on $s$ and the \nameref{rat:t:props:transitive} we deduce that $r + s > 0$.

	Now using the \nameref{rat:t:props:multiplication_order} we get $rs > 0 \cdot s$. From Part (\ref{rat:e:6:3}) of this exercise and the \nameref{rat:t:props:commutative_mult} we see that $0 \cdot s = s \cdot 0 = 0$, and hence $rs > 0$.
\end{proof}

\begin{proof}[(\ref{rat:e:6:5})]
	Suppose that $r > 0$. By Part (\ref{rat:e:6:1}) of this exercise we know that $0 < 1$. From the \nameref{rat:t:props:multiplication_order} it follows that $0 \cdot r < 1 \cdot r$. Because of the \nameref{rat:t:props:commutative_mult} we see that $1 \cdot r = r \cdot 1$, so $0 \cdot r < r \cdot 1$. But then from hypothesis on $r$ and the Lemma \ref{rat:l:alt_def} (\ref{rat:l:alt_def:6}) we deduce that $\frac{0}{r} < \frac{1}{r}$. Finally, by Definition \ref{rat:d:div} we observe that $\frac{0}{r} = 0$, and hence $\frac{1}{r} > 0$.
\end{proof}

\begin{proof}[(\ref{rat:e:6:6})]
	Suppose that $0 < r < s$. From the \nameref{rat:t:props:transitive} we see that $s > 0$. By Part (\ref{rat:e:6:5}) of this exercise we deduce that $r^{-1} > 0$ and $s^{-1} > 0$. By repeated use of the \nameref{rat:t:props:multiplication_order} we obtain $r r^{-1} s^{-1} < s r^{-1} s^{-1}$. From the \nameref{rat:t:props:commutative_mult} we observe that $s r^{-1} = r^{-1} s$, so using the \hyperref[rat:t:props:associative_mult]{Associative} and \hyperref[rat:t:props:inverses_mult]{Inverses} Laws for Multiplication we get $1 \cdot s^{-1} < r^{-1} \cdot 1$. Finally, by the \hyperref[rat:t:props:commutative_mult]{Commutative} and \hyperref[rat:t:props:identity_mult]{Identity} Laws for Multiplication we can conclude that $\frac{1}{s} < \frac{1}{r}$.
\end{proof}

\begin{proof}[(\ref{rat:e:6:7})]
	Suppose that $0 < r < p$ and $0 < s < q$. We note that $p > 0$ because of the \nameref{rat:t:props:transitive}. Now we obtain $r s < p s$ and $s p < q p$ according to the \nameref{rat:t:props:multiplication_order}. From the \nameref{rat:t:props:commutative_mult} we deduce that $p s < p q$. But then $r s < p s < p q$, and hence by the \nameref{rat:t:props:transitive} we can conclude that $r s < p q$.
\end{proof}


%-------------------------------------------------------------------------------------------------------------
\Newpage
\begin{exercise} %% 1.5.7
	\label{rat:e:7}
	\hfill
	\begin{enumerate}
		\item Prove that $1 < 2$. \label{rat:e:7:1}
		\item Let $s, t \in \mathbb{Q}$. Suppose that $s < t$. Prove that $\frac{s + t}{2} \in \mathbb{Q}$, and that $s < \frac{s + t}{2} < t$. \label{rat:e:7:2}
	\end{enumerate}
\end{exercise}

\begin{proof}[(\ref{rat:e:7:1})]
	By Exercise \ref{rat:e:6} (\ref{rat:e:6:1}) we have $0 < 1$. Because of the \nameref{rat:t:props:addition_order} we obtain $0 + 1 < 1 + 1 = 2$. Using the \hyperref[rat:t:props:commutative_add]{Commutative} and \hyperref[rat:t:props:identity_add]{Identity} Laws for Addition we see that $0 + 1 = 1 + 0 = 1$, and hence $1 < 2$.
\end{proof}

\begin{proof}[(\ref{rat:e:7:2})]
	By Exercise \ref{rat:e:6} (\ref{rat:e:6:1}) and Part (\ref{rat:e:7:1}) of this exercise we know that ${0 < 1 < 2}$, so using the \nameref{rat:t:props:transitive} we deduce that $2 > 0$, and as a result $2 \in \mathbb{Q}^{*}$. Hence, ${\frac{s + t}{2} \in \mathbb{Q}}$.

	It follows from Exercise \ref{rat:e:6} (\ref{rat:e:6:5}) that $2^{-1} > 0$, and from the \nameref{rat:t:props:multiplication_order} we can conclude that $s \cdot 2^{-1} < t \cdot 2^{-1}$.

	From the \nameref{rat:t:props:addition_order} we get $s \cdot 2^{-1} + s \cdot 2^{-1} < t \cdot 2^{-1} + s \cdot 2^{-1}$. Because of the \nameref{rat:t:props:identity_mult} we observe that $s \cdot 2^{-1} + s \cdot 2^{-1} = (s \cdot 2^{-1}) \cdot 1 + (s \cdot 2^{-1}) \cdot 1$. Also, from the \nameref{rat:t:props:associative_add} and the \nameref{rat:t:props:commutative_mult} we observe that $t \cdot 2^{-1} + s \cdot 2^{-1} = 2^{-1} \cdot s + 2^{-1} \cdot t$, so using the \nameref{rat:t:props:distributive} we deduce that $s \cdot 2^{-1}(1 + 1) = s \cdot 2^{-1} \cdot 2 < 2^{-1}(s + t) = (s + t)2^{-1}$. From the \hyperref[rat:t:props:commutative_mult]{Commutative} and  \hyperref[rat:t:props:inverses_mult]{Inverses} Laws for Multiplication it follows that $2^{-1} \cdot 2 = 2 \cdot 2^{-1} = 1$. By the \hyperref[rat:t:props:associative_mult]{Associative} and \hyperref[rat:t:props:identity_mult]{Identity} Laws for Multiplication it follows that $s \cdot 2^{-1} \cdot 2 = s \cdot (2^{-1} \cdot 2) = s \cdot 1 = s$. Hence, $s < \frac{s + t}{2}$.

	By the \nameref{rat:t:props:addition_order} we obtain $s \cdot 2^{-1} + t \cdot 2^{-1} < t \cdot 2^{-1} + t \cdot 2^{-1}$. From the \nameref{rat:t:props:commutative_mult} and the \nameref{rat:t:props:distributive} we deduce that ${(s + t)2^{-1} < t (1 + 1)2^{-1} = t \cdot 2 \cdot 2^{-1}}$. We already know that $2 \cdot 2^{-1} = 1$, so using the \hyperref[rat:t:props:associative_mult]{Associative} and \hyperref[rat:t:props:identity_mult]{Identity} Laws for Multiplication we get $t \cdot 2 \cdot 2^{-1} = t(2 \cdot 2^{-1}) = t \cdot 1 = t$. Hence, $s < \frac{s + t}{2} < t$.
\end{proof}


%-------------------------------------------------------------------------------------------------------------
\Newpage
\begin{exercise} %% 1.5.8
	\label{rat:e:8}
	Let $r \in \mathbb{Q}$. Suppose that $r > 0$.
	\begin{enumerate}
		\item Prove that if $r = \frac{a}{b}$ for some $a, b \in \mathbb{Z}$ such that $b \neq 0$, then either $a > 0$ and $b > 0$, or $a < 0$ and $b < 0$. \label{rat:e:8:1}
		\item Prove that $r = \frac{m}{n}$ for some $m, n \in \mathbb{Z}$ such that $m > 0$ and $n > 0$. \label{rat:e:8:2}
	\end{enumerate}
\end{exercise}

\begin{proof}[(\ref{rat:e:8:1})]
	Suppose that $r = \frac{a}{b}$ for some $(a, b) \in \mathbb{Z} \times \mathbb{Z}^{*}$. Because of Exercise\,\ref{rat:e:6}\,(\ref{rat:e:6:1}) we know that $1 > 0$, and we note that $\frac{0}{1} = 0$ according to Definition \ref{rat:d:div}. Taking into account Lemma \ref{rat:l:alt_def} (\ref{rat:l:alt_def:6}) we consider the following two cases.
	\begin{bycases}
		\item $b < 0$. Then $\frac{0}{1} < \frac{a}{b}$ implies $0 \cdot b > 1 \cdot a$, and hence $a < 0$.
		\item $b > 0$. Then $\frac{0}{1} < \frac{a}{b}$ implies $0 \cdot b < 1 \cdot a$, and hence $a > 0$.
	\end{bycases}
\end{proof}

\begin{proof}[(\ref{rat:e:8:2})]
	From Definition \ref{rat:d:div} and Theorem \ref{rat:t:int} (\ref{rat:t:int:div}) we observe that there is ${(a, b) \in \mathbb{Z} \times \mathbb{Z}^{*}}$ such that $r = \frac{a}{b}$. By Part (\ref{rat:e:8:1}) of this exercise we deduce that $a$ and $b$ are either both positive or both negative. If $a$ and $b$ are both positive, then we will clearly reach the desired result. Now suppose that $a < 0$ and $b < 0$. Let $m = -a$ and $n = -b$, which means that $m$ and $n$ are both positive. Using Exercise \ref{rat:e:3} (\ref{rat:e:3:2}) we can notice that $1 = \frac{-1}{-1}$. Then,
	$$
		r = \frac{a}{b} = \frac{a}{b} \cdot 1 = \frac{a}{b} \cdot \frac{-1}{-1} = \frac{a(-1)}{b(-1)} = \frac{-a}{-b} = \frac{m}{n},
	$$
	as required.
\end{proof}


%-------------------------------------------------------------------------------------------------------------
\Newpage
\begin{exercise}[Used in Lemma \ref{cuts:l:aprops} and Exercise \ref{cuts:e:2}] %% 1.5.9
	\label{rat:e:9}
	Let $r, s \in \mathbb{Q}$.
	\begin{enumerate}
		\item \label{rat:e:9:1}
		      Suppose that $r > 0$ and $s > 0$. Prove that there is some $n \in \mathbb{N}$ such that $s < n r$.

		      \hfill[Use Exercise \ref{rat:e:6}, \ref{rat:e:8}, and either Exercise \ref{int:e:8} or \ref{int2:e:4}.]
		\item \label{rat:e:9:2}
		      Suppose that $r > 0$. Prove that there is some $m \in \mathbb{N}$ such that $\frac{1}{m} < r$.
		\item \label{rat:e:9:3}
		      For each $x \in \mathbb{Q}$, let $x^{2}$ denote $x \cdot x$. Suppose that $r > 0$ and $s > 0$. Prove that if $r^{2} < s$, then there is some $k \in \mathbb{N}$ such that $\left( r + \frac{1}{k} \right)^{2} < s$.
	\end{enumerate}
\end{exercise}

\begin{proof}[(\ref{rat:e:9:1})]
	Using Exercise \ref{rat:e:8} (\ref{rat:e:8:2}) we then deduce that $r = \frac{a}{b}$ and $s = \frac{c}{d}$ for some ${a, b, c, d \in \mathbb{Z}^{+} - \{ 0 \}}$. Then $a d > 0$ and $b c > 0$ because of Exercise \ref{rat:e:6} (\ref{rat:e:6:4}), and then $a d \geq 1$ and $b c \geq 1$ because of Exercise \ref{int:e:8}. This means that $b c \in \mathbb{N}$ according to Exercise \ref{int2:e:4}.

	By Exercise \ref{rat:e:6} (\ref{rat:e:6:1}) we know that $0 < 1$, so we obtain $b c < b c + 1$. Let $n = b c + 1$. Then $n \in \mathbb{N}$ and $b c < n$. If $a d = 1$ then $b c < n \cdot 1 = n (a d)$, and by Lemma\,\ref{rat:l:alt_def}\,(\ref{rat:l:alt_def:1}) we have $s = \frac{c}{d} < n \cdot \frac{a}{b} = n r$, as required. Now suppose that $a d > 1$. From Exercise \ref{rat:e:6} (\ref{rat:e:6:7}) it then follows that $1 \cdot b c < a d (b c + 1)$, so $b c < n (a d)$. Using Lemma \ref{rat:l:alt_def} (\ref{rat:l:alt_def:1}) we see that $\frac{c}{d} < n \cdot \frac{a}{b}$, and hence $s < n r$.
\end{proof}

\begin{proof}[(\ref{rat:e:9:2})]
	Because of Exercise \ref{rat:e:8} (\ref{rat:e:8:2}) we have $r = \frac{a}{b}$ for some $a, b \in \mathbb{Z}$ such that $a > 0$ and $b > 0$. We know that $a, b \in \mathbb{Q}$, so using Part (\ref{rat:e:9:1}) of this exercise we can find some $m \in \mathbb{N}$ such that $b < m a$. Then $1 \cdot b < m a$, and hence by Lemma \ref{rat:l:alt_def} (\ref{rat:l:alt_def:1}) we deduce that $\frac{1}{m} < \frac{a}{b} = r$.
\end{proof}

\begin{proof}[(\ref{rat:e:9:3})]
	Suppose that $r^2 < s$. Then $s - r^2 > 0$. From Exercise \ref{rat:e:6} (\ref{rat:e:6:1}) we know that $1 > 0$. By repeated use of Exercise \ref{rat:e:6} (\ref{rat:e:6:4}) we see that $r + r + 1 = 2r + 1 > 0$. It then follows from Exercise \ref{rat:e:6} (\ref{rat:e:6:5}) that $\frac{1}{2r + 1} > 0$, and hence by Exercise \ref{rat:e:6} (\ref{rat:e:6:4}) we deduce that $\frac{s - r^2}{2r + 1} > 0$.

	By Part (\ref{rat:e:9:2}) of this exercise we can find some $k \in \mathbb{N}$ such that $\frac{1}{k} < \frac{s - r^2}{2r + 1}$. We note that $k \in \mathbb{Z}$ and $\frac{1}{k} > 0$ according to Exercise \ref{int2:e:4} and Exercise \ref{rat:e:6} (\ref{rat:e:6:5}). Because of Lemma\,\ref{rat:l:alt_def}\,(\ref{rat:l:alt_def:6}) we have $1 \cdot (2r + 1) = 2r + 1 < k(s - r^2)$. Multiplying both sides of this inequality by $\frac{1}{k}$ it follows that
	$$
		\frac{1}{k}(2r + 1) = 2r \frac{1}{k} + \frac{1}{k} < s - r^2 = \frac{1}{k} k (s - r^2).
	$$

	By Exercise \ref{int:e:8} we observe that $k \geq 1$, so $k^2 \geq k$. Using Exercise \ref{rat:e:6} (\ref{rat:e:6:6}) we get $\frac{1}{k^2} = \left( \frac{1}{k} \right)^2 \leq \frac{1}{k}$. Then
	$$
		2r \frac{1}{k} + \left( \frac{1}{k} \right)^2 \leq 2r \frac{1}{k} + \frac{1}{k} < s - r^2,
	$$
	and then $2r \frac{1}{k} + \left( \frac{1}{k} \right)^2 < s - r^2$. Finally, adding $r^2$ to this inequality and doing some rearranging we obtain
	$$
		r^2 + 2 \frac{1}{k} r + \left( \frac{1}{k} \right)^2 = \left( r + \frac{1}{k} \right)^2 < s,
	$$
	as required.
\end{proof}